\documentclass[11pt]{article}
\usepackage[english]{babel}
\usepackage[normalem]{ulem}

\usepackage{fourier}
\usepackage{color,soul}

\usepackage{url}
\usepackage{color}
\newcommand{\be}{\begin{enumerate}}
\newcommand{\ee}{\end{enumerate}}
\newcommand{\german}[1]{{\color{red}#1$^{de}$}}
\newcommand{\french}[1]{{\color{red}#1$^{fr}$}}

\newcommand{\highlight}[1]{{\color{blue}#1}}


% The first line doesn't show \orig, \dd, \dcom
%\newcommand{\orig}[1]{}
\newcommand{\orig}[1]{{\color{red} {#1}}}

% \dd will be used for suggestion to alter the text/examples; \orig will be used for the original text, \dcom are comments/questions addressed to Omri
%\newcommand{\dd}[1]{}
\newcommand{\dd}[1]{{\color{blue}{#1}}}
\newcommand{\dn}[1]{{\color{green}{#1}}}

%\newcommand{\dcom}[1]{}
\newcommand{\dcom}[1]{\textit{\color{blue}{#1}}}

\newcommand{\dout}[1]{}


\newcommand{\discuss}[1]{{\color{green}{Discuss: #1}}}
\newcommand{\oa}[1]{{\color{blue}{OA: #1}}}
\newcommand{\nss}[1]{{\color{magenta}{NSS: #1}}}
%\newcommand{\nss}[1]{}
\newcommand{\todo}[1]{\footnote{\bf TODO: \color{blue}{#1}}}
\newcommand{\rem}[1]{{(\it #1})}


\usepackage[top=0.5in, bottom=1.5in, left=0.5in, right=0.5in]{geometry}


\setcounter{tocdepth}{4}

\title{UCCA's Foundational Layer:  Annotation Guidelines v2\\
\url{http://www.cs.huji.ac.il/~oabend/ucca.html}}
\begin{document}
\maketitle
\tableofcontents

\newpage


\section{General Principles for Annotation}

\begin{enumerate}

\item
  A UCCA annotation task consists of the annotation of multiple sentences, usually a 
  paragraph or several paragraphs long.
  When you receive a task, take a few minutes to read the entire text, in order to 
  understand the context.
    

 \item
  UCCA divides the text into units (stretches of text; not necessarily contiguous), each referring to a relation,
  a participant in a relation or a relation along with its participants. The types of relations we annotate
  are listed below.

%\dcom{This is a bit unclear. We first say that text is a sequence of scenes and then say that it is a sequence of relations or participants. Does it mean that the sequence of scenes can be easily reconstructed from the annotations?

%Moreover, a bit later we say

%\textit{Each Scene is considered a unit, but there is no one category for Scenes. Instead, the category of the Scene unit reflects the role of that unit in the super-ordinate relation it participates in (see below).}

%Which seems to hold true only for some Scenes as first-order Scenes do not receive any category at all and are strictly implicit.

%[Some thoughts after reading more: at some point we say explicitly that Scenes are evoked exclusively by S/Ps. I suggest that as much as possible be explained using S/Ps directly and without using the notion of Scene because we do not mark for them in most cases and they are rather abstract.]}  

\item
  The units must cover all the tokens, except punctuation tokens which are not annotated.
  
\item
  Units may contain other sub-units, giving rise to a hierarchical structure.
  
\item
  Each unit is assigned a category, reflecting its role in a super-ordinate relation in which it participates. The category does not necessarily reflect the meaning of the unit taken in itself. For instance, all the units in {\bf boldface} have the same category, as they all describe ``horse'' in finer detail:

\begin{itemize}
\item
  ``A {\bf beautiful} horse''
\item
  ``A {\bf police} horse''
\item
  ``A horse {\bf with no name}''
\item
  ``The horse {\bf that won the race}''
\item
  ``A {\bf winning} horse''
\end{itemize}

\item
  UCCA does not annotate ambiguity. 
  When reading ambiguous text, decide on the most likely interpretation in your opinion and use it for annotating
  the entire passage.

\end{enumerate}


\section{A Bird's Eye View of the Categories}


Units may be analyzed according to {\bf one} of the following models:

\subsection*{Model \#1: Scenes}\label{model1}



\begin{enumerate}
  \item
    A Scene is some description of an action, movement or state (including abstract actions or states).
    It generally has a time when it happened, a location, and a ground (the circumstances in which the description was
    uttered or written).
    In concrete cases, a Scene can be imagined as a single mental image or a short script.
    \begin{itemize}
    \item
      ``Woody walked in the park'' (1 Scene)
    \item
      ``I got home and took a shower'' (2 Scenes)
    \end{itemize}

  \item
    A Scene has a main relation (exactly one), which determines the type of the Scene and what happened in it. This main relation can be either a {\sc State} (S) -- if the Scene is constant in time; or a {\sc Process} (P) -- an action, movement or some other relation that evolves in time.
     
  \item
    %\orig {
    %Each Scene is considered a unit, but there is no one category for Scenes.
    %Instead, the category of the Scene unit reflects the role of that unit in the
    %super-ordinate relation it participates in (see below).}
    Each Scene is considered a unit, and is therefore, like all units, also assigned a category as a whole. The category of the Scene unit reflects the role of that unit in the super-ordinate relation it participates in (see below).

  \item
Scenes may contains any number of {\sc Participants} (A).  These are the principal participants in the main relation of the Scene (including locations). Participants may refer either to physical or abstract entities.

  \begin{itemize}
  \item
    ``John$_A$ boiled [an egg]$_A$''
  \item
    ``Programming$_A$ is widely taught nowadays''
  \end{itemize}

  \item
    In static Scenes, the main relation is annotated as a {\sc State}. The State unit should not include its auxiliary verbs if present.

    \begin{itemize}
    \item
      ``John$_A$ is tall$_S$''
    \item
      ``[The apple tree]$_A$ is in$_S$ [the garden]$_A$''
    \item
      ``[An apple]$_A$ weighs$_S$ [200 g]$_A$''
    \item
      ``[This apple]$_A$ does weigh$_S$ [200 g]$_A$''
    \item
      ``John$_A$ is [a writer]$_S$"
    \item
      ``[Big$_S$ \rem{dogs}$_A$]$_E$ dogs$_C$'' 
    \end{itemize}
    
    %\dcom{Flagging a possible issue here: in Irish, it possible to say ``John is a student'' by saying ``John is in his student'', which kills the difference between ``to be in X'' and ``to be X''. Is there a way to make ``writer'' an A as well?}
    
  \item
    In dynamic Scenes, the main relation is marked as a {\sc Process} (P). The Process unit should not include its auxiliary verbs if present.
    
    \begin{itemize}
    \item
      ``John$_A$ kicked$_P$ [the ball]$_A$''
    \item
      ``John$_A$ has been kicking$_P$ [the ball]$_A$ since breakfast''
    \end{itemize}
    
  \item
    {\sc Adverbials} (D) are relations that do not introduce another Scene, but semantically modify the Scene or its {\sc Process} or {\sc State}. Common cases of Ds are modal relations (like ``can'', ``may'' or ``want''), manner relations (like ``quickly'' or ``patiently'') and relations that specify a sub-event (like ``begin'' or ``finish'').
    
    \begin{itemize}
    \item
      ``John$_A$ began$_D$ swimming$_P$''
    \item
      ``John$_A$ may$_D$ come$_P$ [to the party]$_A$''
    \item
      ``[His workers]$_A$ treat$_P$ him$_A$ [with disrespect]$_D$''
    \item
      ``John$_A$ cleverly$_D$ answered$_P$ [the manager's question]$_A$''
    \end{itemize}
    
    See Chapter \ref{app:AD-distinction} for how to distinguish Ds and As in edge cases.

  \item
    Units whose primary purpose is to specify the time in which the Scene occurred should be marked as {\sc Time} (T). However, if time is described by introducing another Participant or another Scene, it should receive a Scene or Participant category instead. Frequency and duration are also considered {\sc Time}.
    
    \begin{itemize}
    \item ``John$_A$ may$_D$ come$_P$ later$_T$''
    \item ``John$_A$ may$_D$ come$_P$ [at around eight]$_T$''
	\item ``I$_A$ get$_F$ treated$_P$ regularly$_T$''
	\item ``[John$_A$ [showed up]$_P$]$_H$ during$_L$ [[the$_E$ filming$_C$]$_P$]$_H$'' (two Scenes! see below)
    \end{itemize}
    
  
  
\end{enumerate}


%%%%%%%%%%%%%%%%%%%%%%%%%%%%%%%%%%%%%%%%%%%%%%%%%%%%%%%%%%%
\subsection*{Model \#2: Non-Scene Units}\label{model2}

%\dd {discuss: the order of the items in this subsection. Maybe it would be clearer to start with Center, then Elaborator and only then the other Relations. Alternative phrasing for example: 
%"In some cases a unit can be internally analyzed, but its elements do not evoke a scene. We distinguish between several types of non-Scene relations. The Center (C) is the main entity in a unit that does not evoke a Scene. Metal Belt$_C$ Queen Elizabeth$_C$
%Elaborators (E) add some information on that main entity ..." {\bf OMRI:} OK. Why won't you give it a try}


Some relations do not evoke a Scene on their own, and give rise to {\it Non-Scene units}. We distinguish between several types of such units.

%\dcom{How does this tie in with the fact that a text is a sequence of Scenes? Does this say essentially that Scenes are not infinitely recursive? I think the hierarchy should be described somewhere.}

\begin{enumerate}

\item
 {\sc Connectors} (N) relate two or more entities (annotated as Centers) in a way that 
 highlights the fact that they have a similar type or role.
 They are usually conjuncts such as the English ``and'', ``or'', ``instead of'' or ``except'' or the German ``sowie'', ``ebenso'' and ``genauso wie''.

  \begin{itemize}
  \item
    ``[John$_C$ and$_N$ Mary$_C$]$_A$ went$_P$ [to school]$_A$ together$_D$''
  \item
    ``I$_A$ 'll$_F$ have$_P$ [coffee$_C$ and$_N$ cookies$_C$]$_A$'' 
    %\nss{this is one of many places where ] is preceded by a space. if that's not intentional, should be easy to fix with a find-replace.}
  \end{itemize}
  
\item
  {\sc Elaborators} (E) add some information about one specific entity or relation, without changing its semantic type. %, which is not a state. \nss{contradicted by ``big dogs'' example above?} 
  These either include inherent attributes (attributes that cannot change 
  because they define the entity), or one of several types of relations specified 
  below. The unit which is elaborated on, and which determines the type of the 
  unit, is marked as a {\sc Center} (C).
  
  \begin{itemize}
    \item
      ``Queen$_C$ of$_R$ England$_E$'' (describes a type of a queen; the fact that she is the queen of England is inherent to her being a queen)
	\item
	  ``His$_E$ father$_C$''
	\item
      ``Chocolate$_E$ cookies$_C$'', ``Metal$_E$ belt$_C$'' (the substance something is made of)
  \end{itemize}
  
  Other types of relations that are considered Es:

  \begin{itemize}
  \item
    Determiners (``the$_E$ last king$_C$ of Scotland'')
  \item
    Class descriptor: Units comprised of a sub-unit that specifies the name of the entity in question,
    and another unit specifies which category it belongs to.
		In these cases, the specific unit is marked with $C$ and the class descriptor is marked as $E$.
		
		\begin{itemize}
		\item
			``[the name]$_E$ [John]$_C$''
		\item
			``the state$_E$ of Washington$_C$''
		\item
			``[the year]$_E$ [1966]$_C$''
                \end{itemize}
  \end{itemize}

\item
  If it is hard to say which of the sub-units adds information to which, both units should be marked as Cs. That is,
  if there is no one unit that determines the type of entity, all units that determine its type should be marked as Cs.

  A frequent example of that is part-whole relations: Units comprised of a sub-unit that specifies the whole,
  and one that specifies a sub-part of it. 

  \begin{itemize}
  \item
    ``bottom$_C$ of$_R$ [the$_E$ sea$_C$]$_C$''
  \item
    ``tip$_C$ of$_R$ [the$_E$ iceberg$_C$]$_C$''
  \german{\item
    ``See-$_C$ boden$_C$''
  } 
  \end{itemize}

\end{enumerate}

% \item
%   Expressions that are used to describe the quantity or magnitude of an entity are marked as {\sc Quantifiers (Q)}.
%   We mark as Qs also words that set define that an entity is a group or a set (e.g., ``group of ...'', ``hundreds of ...'').

%   \begin{itemize}
%   \item
%     ``three$_Q$ apples$_C$''  

%   \item
%     ``several$_Q$ apples$_C$''
%   \item
%     ``[dozens]$_Q$ of$_R$ journalists$_C$''
%   \item
%     ``[a$_E$ group$_C$]$_Q$ of$_R$ journalists$_C$''
%   \item
%     ``[a$_E$ swarm$_C$]$_Q$ of$_R$ bees$_C$''
%   \item
%     ``throngs$_Q$ of$_R$ fans$_C$''
%     \item
%     ``[a$_E$ variety$_C$]$_Q$ of$_E$ colors$_C$''

%   \end{itemize}
  
%   Units of measurement are marked as Cs, and their quantity is a Qs:
  
%   \begin{itemize}
%     \item
%       ``5$_Q$ meters$_C$''
%     \item
%       ``200$_Q$ grams$_C$''
%   \end{itemize} 
  



%%%%%%%%%%%%%%%%%%%%%%%%%%%%%%%%%%%%%%%%%%%%%%%%%%%%%%%%%%%
\subsection*{Model \#3: Inter-Scene relations}


\begin{enumerate}
  
\item
  Linkage is the term for inter-Scene relations in UCCA. There are four major types of 
  relations in
  which Scenes may participate, and therefore four types of categories Scene units may 
  receive.
  The next three items describe these types.
  
\item
  \textbf{Elaborator Scenes:} an E-Scene adds information to a previously established 
  unit. Usually answers
  a ``which X'' or ``what kind of X'' question. Es should place the C they are 
  elaborating as a {\it remote} A (see below).
  A way to check where a Scene is an E-Scene is to ask whether the Scene along with the C 
  it relates to are
  of the same type as the C itself.

  \begin{itemize}
  \item
    ``[The$_E$ dog$_C$ [that ate my homework \rem{dog}$_A$]$_{E}$ ]$_A$ is brown'' 
    (``dog'' is a remote A in ``that ate my homework'')
  \item
    ``The$_E$ person$_C$ [whom$_F$ I$_A$ gave$_P$ [the present]$_A$ [to$_R$ 	
        \rem{person}$_C$]$_A$ ]$_{E}$''
  \item
    ``Brad played [an$_E$ American$_C$ [taken to the Adriatic 	
    	\rem{American}$_A$]$_{E}$]$_A$''
  \end{itemize}

%%% Omri got here
\item
  \textbf{Participant Scenes:} an A-Scene is a participant in a larger Scene. It does not add 	information to some specific participant in it, and if you remove it, it doesn't retain the same type. Usually answers a ``what'' question about the Scene.
  %\dcom{For \textit{it}s in the sentence; the last one seems to be genuinely ambiguous.}
  
  \begin{itemize}
  \item
    ``[Talking to strangers]$_A$ is$_F$ ill-advised$_S$'' (answers ``what is ill-advised?'')
  \item
    ``John$_A$ said$_P$ [he's hungry]$_A$'' (answers ``what did John say?'')
  \item
    ``[[John$_C$ 's$_R$]$_A$ accurate$_D$ kick$_P$]$_A$ saved$_P$ [the game]$_A$'' (answers ``what saved the game?'')
  \end{itemize}
  
\item
  {\bf Parallel Scenes:} any other Scene receives the category Parallel Scene (H). Sometimes there
  is an accompanying relation word and sometimes not. If so, it is a Linker (L). Note that there are no Adverbial (D) Scenes.
  Except for Ground (see below), if a Scene is not an A (Participant) or an E (Elaborator), it's an H.

  \begin{itemize}
  \item
    ``[John managed to amuse himself]$_H$ while$_L$ [waiting in line \rem{John}$_A$]$_H$''
  \item
		``[My house feels fresh]$_H$ [thanks to]$_L$ [[the Battery Park Pest]$_A$ (IMPLICIT)$_P$]$_H$.''
	\item
    ``[The minute]$_L$ [I got home]$_H$ [I noticed the new painting]$_H$''
  \item
    ``If$_L$ [you build it]$_H$ [they will come (IMPLICIT)$_A$]$_H$'' (where they come to is implicit)
  \item
    ``[I'd done some research (IMPLICIT)$_A$]$_H$, [asked a couple of questions \rem{I}$_A$ (IMPLICIT)$_A$]$_H$ and$_L$ [started thinking \rem{I}$_A$ (IMPLICIT)$_A$]$_H$''(the topic of research, questions and thinking is implicit)
  \item
    ``[You're only saying this]$_H$ because$_L$ [John told you to \rem{say this}$_A$]$_H$'' %\nss{ellipsis---shouldn't this have a remote?}
  \german{\item
    ``Nach$_L$ [einer Rolle in einem Thriller]$_H$ [spielte sie in einem Actionfilm mit]$_H$.''}
  \end{itemize}

  Specific cases of Parallel Scenes include (examples of relevant Linkers in brackets): purposive (``in order to'' or ``to''\german{, `um + zu-Infinitiv''}), logical (``if ... then ...''), temporal (``when X, Y'', ``before X, Y''), coordination (``and'', ``but''), and contrastive linkages (``however'', ``still''\german{, ``jedoch''}).

Scenes that are not related to any other units and are therefore in the top level of organization in the text are also Hs (Parallel Scenes). 
%\dn{How does this help? It only seems to makes the notion of ``parallel'' more vague. (A sequence of parallel Scenes?)}\oa{what alternative do you suggest?} \dn{DN: I mean, do we really need this note? Things that are at the top level are unmarked, as fas as I understand. More broadly, I see the motivation for the `parallel' from \textit{while}-clauses and such, but maybe `independent scene' would be a better term.}

  Linkers do not necessarily appear between the Scenes they are linking (see example \#2 above).

\item
  A unit is marked as {\sc Ground} (G), if its primary purpose is to relate some unit to its 	the speech event; either the speaker, the hearer or the general context in which the text 	 was uttered/written/conceived.\footnote{The speech event is called Ground following R. Langacker.}
  %\orig {Gs are used for annotating mentions of the ground Scene that are missing almost all 
  %their elements except for one word or expression.}

  Gs are similar to Ls, except that they don't relate the Scene to some other Scene in the text, but rather to the speech act
  of the text (the speaker, the hearer or their opinions). By convention,
  Ground units should be positioned within the Scene they relate to.

  \begin{itemize}
  \item
    ``[Surprisingly$_G$ , [our flight]$_A$ arrived$_P$ [on time]$_T$]$_H$''
  \item
    ``[[In my opinion]$_G$, John$_A$ is$_F$ coming$_P$ home$_A$ ]$_H$'' 
%\dn{What is the purpose of Hs here? If we only wish to show that there are no parallel scenes here, their absence, as opposed to the example below, will be enough. Otherwise it seems that whenever there is a G, H becomes obligatory.}\oa{This is again the same issue; what do you suggest? calling the top-level elements TOP or just not annotating them entirely?} \dn{DN: You mean that people annotate \textit{passages} and somehow have to separate them into things UCCA treats as a unit? I would suggest to leave these units unmarked and provide a simple way to demarcate boundaries. It seems that scenes never encompass several sentences, so perhaps automatic tokenisation may help? (With manual corrections if needed.)}
  \end{itemize}
  
Note that a complete Scene that refers to the ground (with As and Ds etc.)\ should be annotated as a Scene and not as a G. That is, if a unit alludes to the speech event, but is missing almost all its elements save for one word or expression, it should be a G. If the speech event is mentioned more elaborately, it should be annotated as a Scene.

  \begin{itemize}
  \item
    ``[I$_A$ was$_F$ surprised$_S$ ]$_H$ when$_L$ [[our flight]$_A$ arrived$_P$ [on time]$_T$]$_H$''
  \item
    But: ``[Surprisingly$_G$, [our flight]$_A$ arrived$_P$ [on time]$_T$]$_H$''
  \item
    ``I$_A$ told$_P$ you$_A$ already$_D$ [that John can't make it]$_A$''
  \end{itemize}
  
\end{enumerate}


%%%%%%%%%%%%%%%%%%%%%%%%%%%%%%%%%%%%%%%%%%%%%%%%%%%%%%%%%%%
\subsection*{Categories that Appear in All Models}

There are three types of categories that may appear anywhere in the text: Functions (F), Relators (R) and Quantifiers (Q).

%\dcom{Following is a suggested rephrasing of "Relators", see the orig below (note that some comments were added to the examples in the original version, to see them go over the original): }

\begin{enumerate}

\item 
Relators are relations that relate between two or more entities within Scene units as well as non-Scene units. Rs in English are usually prepositions (see Section \ref{sec:relators} below for a more elaborate discussion).

\begin{itemize}
    \item
Relators within Scene units.
When a Relator connects between the main event (P/S) and another Scene element (A,T,D) then it should be included inside the Scene element (A,T,D) it pertains to.

\begin{itemize}
    \item
      ``John said [that$_R$ he$_A$ 's$_F$ going$_P$ home$_A$]$_A$'' 
    \item
      ``I$_A$ referred$_P$ [to$_R$ John$_C$ and$_N$ Mary$_C$]$_A$''
    \item
      ``I$_A$ referred$_P$ [to$_R$ John$_C$ and$_N$ to$_R$ Mary$_C$]$_A$'' 
\end{itemize}
 
    \item
Within non-Scene units
By convention, we place the Rs in non-Scene units as siblings of the Es, Qs and Cs they relate (on the same level with them). 

\begin{itemize}
    \item
	  ``[a$_E$ group$_C$]$_Q$ of$_R$ journalists$_C$''
	\item
      ``bottom$_C$ of$_R$ [the$_E$ sea$_C$]$_C$''
    \item
      ``Queen$_C$ of$_R$ England$_E$''
  \end{itemize}
  
    \item
	When will we not use Relators?: 
    \begin{itemize}
    \item 
    To link between Parallel Scenes (for that see Linkers). 
    \item
    To connect between Centers that have the same parent unit and carry a similar type or role (for that see Connectors). 
    \end{itemize}

\end{itemize}




% \orig {

% Relators (R) are relations that relate one or more entities without evoking a new Scene. Rs in English are
%   usually prepositions (see Section \ref{sec:relators} below for a more elaborate discussion). They have two main varieties:

%   \begin{itemize}
%     \item
%     Rs can pertain to a specific entity (much like Elaborators), but unlike Es they then relate that entity to
%     other relations/entities in the context. In this case, the R should be positioned as a sibling of the C
%     (or the A Scene) they pertain to.

%     \begin{itemize}
%     \item
%       ``John said [that$_R$ he$_A$ 's$_F$ going$_P$ home$_A$]$_A$'' \dcom {R seems to be a child of an A Scene here.}
%   \end{itemize}
    
%   \item
%     Rs can relate two or more entities within a non-Scene unit
%     (much like Ns). Unlike Ns, they relate entities that are not viewed as similar in type or role.
%     By convention, we place the Rs as siblings of the Es and Cs they relate.
    
%     \begin{itemize}
%     \item
%       ``[a$_E$ group$_C$]$_E$ of$_R$ journalists$_C$'' \dcom{``a group'' is described as Q above.}
%     \item
%       ``bottom$_C$ of$_R$ [the$_E$ sea$_C$]$_C$''
%     \end{itemize} 

% \end{itemize}}


\item
  Functions (F) are units that do not introduce a new participant or relation.
  They can only be interpreted as part of a larger construction in which they are situated,
  or convey some aspect of meaning which is not covered by the foundational layer (e.g., tense or focus).
  Usually in these cases, they cannot be substituted with any other word.

  \begin{itemize}
  \item
    ``I$_A$ want$_D$ to$_F$ run$_P$ [a$_E$ marathon$_C$]$_A$''
  \item
    ``I$_A$ am$_F$ going$_P$ [to$_R$ the$_E$ supermarket$_C$]$_A$''
  \item
    ``It$_F$ is likely$_S$ [that he will make it]$_A$''
  \item
    ``Let$_F$ me$_A$ introduce$_P$ John$_A$''
  \end{itemize}
  
  %\dd {included Fillers in "Utterances"}
  %Fillers (e.g. "ummm", "like", "okay", "so") should be marked also as Fs.
  %(``ummm$_F$ I$_A$ heard$_P$ [you$_A$ say$_P$ that$_A$ ]$_A$'')
  
  %\nss{should ``you say that'' be an $A$ of `heard'?} \oa{well, if that what he heard then yes; it depends on context.} \nss{What is the other interpretation?}
  %\dd {I also think there's a mistake in the original: there are two unrelated Ps on the same level. I agree it should be corrected to "ummm$_F$, I$_A$ heard$_P$ [you say that]$_A$." I can also see two interpretations ("I heard that you generally say that", or "I actually heard you saying it"), but I think this option can  capture them both, no?".}
  
 \item We use {\sc Quantifiers (Q)} to mark expressions that:
 
\begin{itemize}
\item Describe the quantity or magnitude of an entity:\footnote{Expressions of distance will be marked as As and then internally, Quantity to mark the actual measurement: ``He ran [100$_Q$ meters$_C$]$_A$''.}

\begin{itemize}
\item
``three$_Q$ apples$_C$''
\item
``several$_Q$ apples$_C$''
\item
``I$_A$ bought$_P$ [[three$_E$ kilos$_C$]$_Q$ of$_R$ apples$_C$]$_A$''
\end{itemize}
 
\item Any expression that set defines that an entity is a group or a set (e.g., ``group of ...'', ``hundreds of ...'').

\begin{itemize}
\item  
``[a$_E$ group$_C$]$_Q$ of$_R$ journalists$_C$''
\item 
``[a$_E$ swarm$_C$]$_Q$ of$_R$ bees$_C$''
\item  
``[a$_E$ variety$_C$]$_Q$ of$_E$ colors$_C$''
   \end{itemize}

\item In Scene units, for any expression that indicates the amount of occurrences of a single event:

\begin{itemize}
\item
``We$_A$ had$_F$ talked$_P$ [three$_E$ times$_C$]$_Q$ [over the last week]$_T$''
\end{itemize}
 
\item Ordinals are also marked as Qs:

\begin{itemize}
\item
``[My$_A$ first$_Q$ kick$_P$]$_A$ saved$_P$ [the game]$_A$''
\item
``The$_E$ first$_Q$ king$_C$ of$_R$ Scotland$_E$''
\item
``I$_A$ got$_P$ here$_A$ first$_Q$''
\item
``I$_A$ was$_F$ [the first]$_Q$ to$_F$ arrive$_P$''
\end{itemize}
\end{itemize}

\end{enumerate}

%%%%%%%%%%%%%%%%%%%%%%%%%%%%%%%%%%%%%%%%%%%%%%%%%%%%%%%%%%%
\subsection*{Remote and Implicit Units}

\begin{enumerate}
  
\item

There are instances where a sub-unit in a given unit is not explicitly mentioned. We can indicate the missing sub-unit in two ways:
\be
\item Add a reference of the missing unit from another place in the text, as a Remote unit.
\item When it does not appear explicitly in any place in the text, add an Implicit unit to stand for the missing sub-unit.  \ee

We add a Remote/Implicit unit whenever there is a element or relation which we think to be strongly present in the conceptualization of the Scene, but is not explicitly mentioned. 


Remote and Implicit units should be assigned relevant categories like any other unit.

%\orig {When some relation (corresponding to a unit X) is clearly described by the text, but either it or one of its arguments is not expressed explicitly in the text, we say the unit X is missing a sub-unit. If the missing entity is overtly referred to in another place in the text by the unit Y, we say that Y is a remote sub-unit of X. If the omitted unit does not appear explicitly in any place in the text, we say X has an implicit unit. Remote and implicit units have categories just like any other unit.

%Add a remote unit whenever you think there is a participant or relation which is strongly present in your conceptualization of the Scene, but is not explicitly mentioned.}


Examples: (target relations (X) underlined, remote units (Y) boldfaced)

\begin{itemize}
\item	``[{\bf John} got home]$_H$ and$_L$ [\underline{took a shower} \rem{John}$_A$]$_{H}$''
\item	``[The {\bf dog} [\underline{I saw last night} \rem{dog}$_A$]$_{E}$ ]$_A$ was$_F$ brown$_S$''
%\item 	``[We$_A$ just$_T$ opened$_P$ (IMPLICIT)$_A$]$_{H}$'' (the thing opened is implicit) \nss{I think ``we'' is metonymic for the institution that opened, such as a show or business. I don't think ``we'' is agentive in the sense of causing something independent to open---if referring to a bottle you'd need to say ``We just opened it.''} \oa{OK. Let's remove this example.}
\item 	``[Sure$_G$ John$_A$ is$_F$ (IMPLICIT)$_P$]$_H$'' (John is what? if it appears explicitly in the previous sentence, it's a remote unit; otherwise it's an implicit unit)
\item   ``[[John$_A$ is$_F$ {\bf tall}$_S$]$_H$, [\underline{Mary$_A$ is$_F$ n't$_D$} \rem{tall}$_S$ ]$_H$''
%\item	``[What$_A$ does$_F$ {\bf John}$_A$ {\bf want}$_D$ (IMPLICIT)$_P$ ?]$_H$ [\underline{[a$_E$ banana$_C$]$_A$} \rem{John}$_A$ \rem{wants}$_D$ (IMPLICIT)$_P$]$_{H}$'' (main verb, such as ``eat'', omitted) \dcom{DN: Can ``want'' never be treated as a main verb?}\oa{it can, maybe we should choose a different example}\dd {maybe instead: 
%\item ``[enjoyed$_P$ spending the weekend with you (IMPLICIT)$_A$]$_H$`` (who enjoyed the weekend isn't explicitly mentioned)
\end{itemize}

\end{enumerate}

%%%%%%%%%%%%%%%%%%%%%%%%%%%%%%%%%%%%%%%%%%%%%%%%%%%%%%%%%%%
\section{Technical Notes and Guidelines}

\begin{enumerate}
\item
With any problem or question, contact the administrator of the project. 
%\orig {If there is uncertainty, mark the unit as ``uncertain''.} 
If there is uncertainty, mark your guess and add ``uncertain''.
\item
When annotating a remote unit, select the minimal possible relevant unit, and not its ancestors.
\item
Top-level annotation (i.e., of units directly below the passage level) should be annotated, wherever possible, according to the Scene model. The only exceptions are cases that do not describe a Scene in any way (such as section titles).
\item
Prefer Ls over Ds, where possible.
\item
Prefer Ls over Gs where possible.
\item
Prefer Ls over Ts where possible.
\item
Prefer Ds over Ts where possible.
\item
Prefer annotating A-Scenes and E-Scenes over Parallel Scenes where possible.
\item
Prefer separating participants from their relations where possible.
\item
%\orig {Use Ds instead of Es inside P/S. More generally, try to avoid complex or long P/S.}
Prefer Ds over a longer P/S with an E inside it. More generally, try to avoid complex or long P/S.

\item
Use Implicit units sparingly and prefer Remote units where possible.
\item
Do not create units only to be used later as a Remote unit. Use existing units instead.
\item
Since morphology in English is very impoverished, we take a pragmatic approach and in our primary layer do not annotate parts of words, but only sets of complete words . In further layers, a sub-unit may cover a part of a word, as long as that part refers to a relation covered by UCCA or to a participant in it.
\item
Function units (Fs) do not refer to a participant or relation and, since the UCCA annotation reflects participation in relations, it is often not clear in what level of the hierarchy an F unit should be placed in. When this occurs, include the F in the deepest unit that stands to reason.
%\dd{I added an alternative item in the Detailed guidelines titled "A Scene within a Scene or two Parallel Scenes":\orig{\item
%Participant Scenes can be detected by replacing them with a non-Scene unit and seeing where the result is a Participant. If the result after the change is an Adverbial or Time, it should be an H.}
\item
Single words are often Scenes as well. This will usually happen where none of the participants is explicitly mentioned.

\begin{itemize}
\item
  ``[The$_E$ [negotiations$_P$]$_C$ ]$_A$ [took place]$_P$ [in$_R$ Rome$_C$]$_A$''
  %\dd {[[The$_E$ negotiations$_C$]$_P$]$_A$}
\item
  ``The$_E$ [available$_S$]$_E$ options$_C$''  
\item
 ``[Crying$_P$ \rem{you}$_A$ ]$_A$ makes$_D$ you$_A$ stronger$_P$''
 \item
 ``[I$_A$ went$_P$ [to$_R$ the$_E$ store$_C$]$_A$]$_H$ for$_L$ [eggs$_A$ \rem{I}$_A$ (IMPLICIT)$_P$]$_H$'' (``for'' is a purposive linker. The implicit P in in the second Scene is for the buying action)%\dcom{``I reached in my coat pocket for the keys''?}
%\item
	%``Amazing$_S$ !''
\end{itemize}


\end{enumerate}



%%%%%%%%%%%%%%%%%%%%%%%%%%%%%%%%%%%%%%%%%%%%%%%%%%%%%%%%%%%%%%%%%%%%%%%%%%%%%%%%%%%%%%%%%%%%
\section{Classification of Prepositions}\label{sec:relators}

Prepositions are in frequent use in English. They include words such as ``in'', ``on'', ``after'', ``with'' and ``under'' \german{ or ``nach'', ``in'' and ``auf'' in German}.
Some prepositions are multi-worded, in which case they are internally annotated as unanalyzble. Examples include ``thanks to'' and ``on top of''.

\begin{enumerate}
   
\item
{\bf Prepositions as Relators:}
\begin {enumerate} 
\item
In Scene units, Relators are included inside the Scene element they pertain to: 
\begin {itemize}
	\item
	``John$_A$ put$_P$ [the$_E$ hat$_C$]$_A$ [on$_R$ the$_E$ shelf$_C$]$_A$''
	\item
	``John$_A$ relied$_P$ [on$_R$ his$_E$ father$_C$]$_A$''
	%	\item
	%``John$_A$ just$_T$ heard$_P$ [of$_R$ his$_A$ re-election$_P$]$_A$''
	\item
	``John$_A$ referred$_P$ [to$_R$ Mary$_C$]$_A$ [in$_R$ his$_A$ dissertation$_P$]$_A$''
    \item
    ''he$_A$ left$_P$ [in$_R$ a$_E$ hurry$_C$]$_D$''
    \item
    ''[His book]$_A$ was$_F$ published$_P$ [in$_R$ 2014$_C$]$_T$"
    \item
    ''John$_A$ will$_F$ visit$_P$ Mary$_A$ [on$_R$ Sunday$_C$]$_T$''
\end {itemize}

\item  In non-Scene units, they are placed on the same level with the Es, Qs and Cs they relate:
    \begin{itemize}
	\item
	``President$_C$ of$_R$ [the$_E$ USA$_C$]$_E$''
	\item
	``The$_E$ finest$_E$ hotels$_C$ of$_R$ [the$_E$ world$_C$]$_E$''
	\item
	``bottom$_C$ of$_R$ [the sea]$_C$''
	\item
	``[a period]$_C$ of$_R$ time$_C$''
	\item	
	``[a group]$_Q$ of$_R$ journalists$_C$''
	\item
	``millions$_Q$ of$_R$ dollars$_C$''
	%\item 
	%``plenty$_Q$ of$_R$ fish$_C$''
	\item 
	``[four$_Q$ episodes$_C$]$_E$ of$_R$ Dallas$_C$''
    \item
    ``books$_C$ about$_R$ [the War]$_E$``
    \item
    ``People$_C$ with$_R$ [red hair]$_E$``
    \item 
    ``Words$_C$ in$_R$ English$_E$``
  
    
\end {itemize}
\end{enumerate}
%\nss{include some non-``of'' examples, like ``people with red hair'', ``the writing on the wall''?} \oa{good idea. Dotan, could you add some?}\dd {added, disabled one of the ''of'' examples}

\item
{\bf Phrasal verbs:} the preposition changes the semantics of the verb in an unpredictable way. In that case the preposition is considered to be a part of the S or P. The P/S together form an unanalyzable unit (as it does not have sub-parts with significant semantic input).

\begin{itemize}
\item
``John$_A$ [gave up]$_P$ [his pension]$_A$''
\item
``John$_A$ let$_{P-}$ Mary$_A$ down$_{-P}$''
\item
``John$_A$ [took]$_{P-}$ Mary$_A$ [up on]$_{-P}$ [her$_A$ promise$_P$]$_A$'' 
%\dcom{``on'' seems to be a part of the verb here; cf. ``accept an offer / take someone up on an offer''}
\end{itemize}

Note that this case does not cover cases where the preposition doesn't change the semantics of the main relation, but is mandatory (``inherent preposition''), such as in ``John is next [to$_R$ Mary$_C$]$_A$'', ``John relies [on$_R$ Mary$_C$]$_A$''. 



\item
{\bf Main relations:} If the preposition is the main relation in the Scene, it is an S. %\nss{can it ever be $P$?} \oa{good question. not sure we ever ran into one...}

\begin{itemize}
\item
``[The apple tree]$_A$ is$_F$ in$_S$ [the garden]$_A$''
\item
``John$_A$ is$_F$ into$_S$ Mary$_A$''
\end{itemize}

\end{enumerate}


%%%%%%%%%%%%%%%%%%%%%%%%%%%%%%%%%%%%%%%%%%%%%%%%%%%%%%%%%%%%%%%%%%%%%%%%%%%%%%%%%%%%
\section{Classification of Copula/Verbless Clauses}\label{sec:copula_verbless}

In some languages, clauses can completely lack verbs (e.g., Hebrew), while in others they would minimally include a copula (e.g., English). Treatment of both cases is similar in UCCA.

\dd{
\begin {enumerate}
\item
{\bf English Copulas}

\begin{enumerate}
\item 
  ''to be'': the verb ''be'' in its different forms will usually be marked F, with the exception of identity cases, in which it will be marked S (for elaboration see the next section)

\begin{itemize}
\item
``[This chair]$_A$ is$_F$ brown$_S$``
\item
``You$_A$ are$_F$ tall$_S$``
\item
``This$_A$ is$_S$ John$_A$ (identity)``
\end{itemize}


\item {\label{seem}}
Verbs of perception and sense: verbs such as seem/look/appear/sound/feel are often used without specifying the experiencer of the feeling/perception. In these cases they should be treated as a G. In case where the experiencer is stated, they should be a separate Scene.

\begin{itemize}
\item 
``[The coffee]$_A$ seems$_G$ to$_F$ be$_F$ hot$_S$``
\item
``[The car]$_A$ looks$_G$ good$_S$``
\item
``It$_F$ seemed$_P$ [to$_R$ Mary$_C$]$_A$ [that$_R$ [the coffee]$_A$ is$_F$ hot$_S$]$_A$``
\item
``It$_F$ appears$_G$ that$_F$ he$_A$ had$_F$ left$_P$ [the country]$_A$``
\item
``[Your dog]$_A$ looks$_G$ like$_F$ [a cat]$_S$``
\end{itemize}


\item
Change-of-state verbs (become, go, get, turn, grow): We treat these as Ds
\begin{itemize}
\item
``John$_A$ grew$_D$ old$_S$``
\item
``Mary$_A$ turned$_D$ ill$_S$``
\end{itemize}

\item
stay, remain, kept: such verbs should be marked D as well
\begin{itemize}
\item
``John$_A$ stayed$_D$ awake$_S$``

\end{itemize}
\end{enumerate}

\item
{\bf Distinguishing identity and other static Scenes.}\label{identity}

Occasionally nouns are used as Ps or Ss, accompanied by some inflection of the verb ``be''. UCCA distinguishes between two cases:
\be \item
Identity: where there are two separate referring expressions that are asserted to have the same referent, defined entities.
\be 
\item
``[The morning star]$_A$ is$_S$ [the evening star]$_A$''
\item
``[That person over there]$_A$ is$_S$ John$_A$''
\ee

\item
Attribution: where there is one A, and the noun is used to describe some set of elements to which that A belongs to. In this case, we include the set-denoting noun in the P or S.
\be \item
"[his speech]$_A$ was$_F$ [an embarrassment]$_S$"
\item
``[Brad Pitt]$_A$ was$_F$ [a slimeball]$_S$ [as the boyfriend in the soap opera Dallas]$_A$''
\item
``John$_A$ is$_F$ [[six$_E$ years$_C$]$_E$ old$_C$]$_S$''
\ee
\ee

\end{enumerate}}


%%%%%%%%%%%%%%%%%%%%%%%%%%%%%%%%%%%%%%%%%%%%%%%%%%%%%%%%%%%%%%%%%%%%%%%%%%%%%%%%%%%%
\section{Participant-Adverbial Distinction}\label{app:AD-distinction}

A basic issue in almost any grammatical theory is to determine when a unit is a participant and when it is a secondary relation. In UCCA, this is the distinction between Participants and Adverbials.

\begin{itemize}
\item
Any unit that introduces a new participant is an A. Subjects, objects, instruments, locations, destinations are therefore invariably As.
\item
Adverbs and any other units that introduce another relation (without introducing a participant) into the Scene are Ds. Manner adverbs (e.g., ``quickly'', ``politely'') are invariably Ds.
\item
Prepositional phrases constitute most of the borderline cases.
\end{itemize}

\noindent
{\bf Examples:}

\begin{enumerate}
\item
``John$_A$ suffered$_P$ [for$_R$ the$_E$ team$_C$]$_A$''
\item
``Woody$_A$ walked$_P$ [in$_R$ the$_E$ park$_C$]$_A$ yesterday$_T$''
\item
``John$_A$ cut$_P$ [the cake]$_A$ [with$_R$ a$_E$ knife$_C$]$_A$''
\item
``John$_A$ behaved$_P$ recklessly$_D$''
\item
``Woody$_A$ treated$_P$ him$_A$ [with$_R$ disrespect$_C$]$_D$''
\item
``Texas$_A$ won$_P$ [in$_R$ its$_E$ home$_E$ court$_C$]$_A$''
\item
``John$_A$ bought$_P$ milk$_A$ [next door]$_A$ [for$_R$ 50$_Q$ p$_C$]$_A$'' (``next door'' is a location, albeit a vague one)
\end{enumerate}


\section{Analyzability}

By default, analyze all cases down to the word level.
The only cases which should not be analyzed are:

\begin{itemize}
\item
Where the internal structure cannot be analyzed using any of the models: Scene, E+C, multiple Cs (possibly with N), inter-Scene relations.
\item
This usually happens where it's not clear what the meanings of the individual words in this context are.
\item
Names should not be internally analyzed.
\end{itemize}

\noindent
{\bf Examples:}

\begin{enumerate}
\item
``The$_E$ October$_E$ [Revolution$_P$]$_C$'': analyzable although it is not
simply a revolution that happened in October, but rather a specific one.
\item
``Chief$_E$ executive$_E$ officer$_C$'': analyzable.
\item
``University$_C$ of$_R$ Texas$_E$'': analyzable.
\item
``[The$_E$]$_P-$ real$_D$ [deal$_C$]$_-P$'':
%``The$_E$ real$_E$ deal$_C$'': 
analyzable although it's an idiomatic expression
since the sub-parts do convey relevant meaning.

\item
``as well as'': unanalyzable since it's not really clear which
categories to assign to the individual words.
\item
``give up'': unanalyzable as it is not clear what meaning ``give'' and ``up'' contribute to the expression.
\item
``I saw Tom Cruise in Top Gun'': ``Tom Cruise'' and ``Top Gun'' are unanalyzable (names). 
\item
``I read The Curious Incident of the Dog in the Night-time'': ``The Curious Incident of the Dog in the Night-time'' is unanalyzable (despite its compositional meaning)
%\nss{What about compositional names (``'', ``Food and Drug Administration'')?} \oa{Names of movies and such we still treat as unanalyzable. We should probably say that.}
\end{enumerate}


\section{Detailed Guidelines}


\subsection{Scenes}

\paragraph{Annotating Scenes within Scenes.} 
In order to analyze a Scene within a Scene we have two options: 

\begin{enumerate}
\item Analyze it first with Center-Elaborator relations (see Section \ref{model2} for elaboration on non-Scene units).
\item Analyze it first with Process/State-Participant relations (see Section \ref {model1} for elaboration on Scene units). 
\end {enumerate}

To determine this we ask ourselves what would we mark as the Center? If it's a concrete entity then we begin with Center-Elaborator relations, but if it's some kind of action or state then we annotate it directly as a Scene.

\begin{itemize}
\item Analysis of a Scene within Scene first with Center-Elaborator relations: 
\begin{itemize}
\item
``[The$_E$ dog$_C$ [that$_R$ ate$_P$ [my homework]$_A$ \rem{dog}$_A$]$_E$ ]$_A$ is$_F$ brown$_S$''
\item
``I$_A$ like$_S$ [ burned$_P$ \rem{coffee}$_A$]$_E$ coffee$_C$]$_A$''
\item
``Brad$_A$ played$_P$ [an$_E$ American$_C$ [\rem{American}$_A$ going$_P$ [to$_R$ the$_E$ Adriatic$_C$ ]$_A$ ]$_E$ ]$_A$''
\end{itemize}

\item Analysis of a Scene within Scene directly as a Scene: 

\begin{itemize}
\item 
``[[John$_C$ 's$_R$]$_A$ kick$_P$] saved$_P$ [the$_E$ game$_C$]$_A$'' 
\item
``John$_A$ said$_P$ [he$_A$ invented$_P$ [skating$_P$]]$_A$'' 
\end{itemize}

\end{itemize}

% \orig{\paragraph{Distinguishing A Scenes and E Scenes.} 

% \be
% \item
% An E scene is when its sibling Center (the sub-unit that determines the semantic type of the whole parent unit) is not a Scene (e.g., some concrete or animate object). The E Scene adds information about the Center, but after the elaboration, the unit still has the same semantic type. \dcom{Can this be recast in terms of Participants and States/Processes? We don't mark for Scenes per se.}

% \be
% \item
% E Scene: ``[The$_E$ dog$_C$ [that$_R$ ate$_P$ [my homework]$_A$ \rem{dog}$_A$]$_E$ ]$_A$ is$_F$ brown$_S$'' (``that ate my homework'' answers the question ``which dog'' and also omitting it ends up with ``the dog'' which is of the same semantic type as ``the dog that ate my homework'').
% \item
% E Scene: ``I$_A$ like$_S$ [ [burned$_P$] $_E$ coffee$_C$]$_A$'' (``burned coffee'' is a type of a coffee. ``burned'' answers the question what kind of coffee. Omitting ``burned'', would end up with ``coffee'', which is not a Scene on its own).
% \item
% E Scene: ``Brad$_A$ played$_P$ [an$_E$ American$_C$ [(American)$_A$ going$_P$ [to$_R$ the$_E$ Adriatic$_C$ ]$_A$ ]$_E$ ]$_A$'' (``an American going to the Adriatic'' is a type of American. Also, ``going to the Adriatic'' answers the question ``what kind of American'').
% \ee

% \item
% An A scene is when the sub-unit that determines the semantic type evokes a Scene. The competing annotation is one where the Center is internally annotated as a Scene. In this case, it's an A-Scene.

% \be
% \item
% A Scene: ``[[John$_C$ 's$_R$]$_A$ kick$_P$] saved$_P$ [the game]$_A$'' 
% \ee

% \item
% In some cases, there is no one sub-unit which has a competing analysis as a Center. Rather, the sub-unit has the regular structure of a Scene. In this case, it's an A-Scene.

% \be
% \item
% A Scene: ``John$_A$ said$_P$ [he$_A$ invented$_P$ [skating$_P$] ($_{REMOTE-A(he)}$ ]$_A$'' (``he invented skating'' is not a type of ``he'' or of ``skating''. Rather it answers the question ``what did John say''. It is therefore an A Scene).
% \dd {I think maybe the Remote unit is redundant here. If you delete it here, it appears in p.13 as well}
% \ee

% \ee}

\paragraph{Dependent Scenes.}
A Scene is not necessarily something that can stand on its own. It may require
a larger construction to rely on, but it is still considered a Scene:
\be
\item
``[he$_A$ retired$_P$]$_H$ [with]$_L$ [ [ [a$_E$ rank$_C$]$_S$ [of$_R$ major$_C$]$_A$ \rem{he}$_A$ ]$_H$''
\item
``[Mary$_A$ cuts$_P$ [her$_E$ hair$_C$]$_A$ ]$_H$ [like]$_L$ [[a$_E$ boy$_C$]$_A$ \rem{cuts}$_P$ ]$_H$''
\item
``[once$_T$ poor$_S$ \rem{he}$_A$]$_H$, [he$_A$ now$_T$ owns$_S$ [a$_E$ spacious$_E$ apartment$_C$]$_A$ ]$_H$''
\ee


\paragraph{Distinguishing Ground and Participant Scenes.}
%\orig{A ground relates the speech event or some aspect of it with a given unit.} 
A Ground unit relates to the speech event itself or some aspect of it. 
It does not introduce a new Scene above and beyond evoking the speech event.
We do not analyze the internal structure of Gs.
\be
\item
``[ [The truth is that]$_G$ John$_A$ is$_F$ [a$_E$ conservative$_C$]$_S$]$_H$''
\item
``[ [Surprisingly]$_G$ , [I]$_A$ [saw]$_P$ [John]$_A$ [in the park]$_A$]$_H$''
\item
``[ [To my surprise]$_G$ [I]$_A$ [saw]$_P$ [John]$_A$ [in the park]$_A$]$_H$''
\item
``[you$_A$ can$_D$ go$_P$ home$_A$, [for all I care]$_G$ ]$_H$''
\ee
Contrast with participant Scenes. Both ``I think'' and ``Mary saw'' introduce a new Scene, with a new P. They are therefore participant Scenes.
\be \item
``I$_A$ think$_P$ [that$_R$ John$_A$ is$_F$ [a$_E$ conservative$_C$]$_S$ ]$_A$''
\item
``Mary$_A$ saw$_P$ [John$_A$ running$_P$ [in$_R$ the$_E$ park$_C$]$_A$ ]$_A$''
\ee

\dd {\paragraph {Speaker attitude  - distinguishing between Ground, Adverbial and State.}

\begin {enumerate}
\item
Any unit that relates to a certain quality of the main event in a Scene should be marked D.

\begin {itemize}
\item
``We had an amazing$_D$ [test drive]$_P$ ! ``
\item
``He ran$_P$ amazingly$_D$ ! ``
\end {itemize}`

\item
Any unit that describes a certain quality of a concrete entity in a Scene should be marked S and the entity - A. 
\begin {itemize}
\item
``Amazing$_S$ book$_A$!``
\end {itemize}

\item
Any unit that expresses the speaker's attitude toward the event, but doesn't directly describe a certain quality of the P/S should be marked G:

\begin{itemize}
\item 
``Amazingly$_G$, we had an excellent time`` (We would have had the same excellent time even if the speaker wasn't amazed)
\item
``Surprisingly$_G$ he went there``
\item
``Interestingly$_G$, he decided to do it`` 
\item 
``They shockingly$_G$ decided to get a divorce``
\end{itemize}
\end {enumerate}}



\paragraph{Static Scenes.} Static Scenes are Scenes which can be fully described by a single picture, that does not change throughout the duration of the state. 
Following are several examples of static Scenes.
\be
\item
Identity. Expressing the identity between two entities.\footnote{Do not confuse identity with ``Noun as a P/S'' (see below).} Identity is the case where there are two well-defined entities (not a set or a relation, but two well-defined entities) and optionally a copula (e.g., ``be''). This is the only case where a copula serves as the main relation of a Scene.
\be
\item
``[The morning star]$_A$ is$_S$ [the evening star]$_A$''
\item
"[That$_E$ person$_C$]$_A$ is$_S$ John$_A$"
\item
But: ``John$_A$ is$_F$ [a$_E$ member$_C$]$_S$ [of$_R$ the$_E$ NRA$_C$]$_A$'' (since here ``a member of the NRA'' is not one specific entity, but a role that could apply to multiple people). %\nss{``set'' is confusing since ``a member'' is singular. How about: ``role that could apply to multiple people''?} \oa{changed}
\ee

\item
Attribution/Benefaction/Location. Specifying a quality, a benefactor or a location of an argument.
\be
\item
``[This$_E$ man$_C$]$_A$ is$_F$ clever$_S$''
\item
``[This$_E$ present$_C$]$_A$ is$_F$ for$_S$ [ [John$_C$ 's$_R$]$_A$ birthday$_P$]$_A$''
\item
``[The$_E$ apple$_E$ tree$_C$]$_A$ is$_F$ in$_S$ [the$_E$ garden$_C$]$_A$''
\ee

\item
Alienable Possession (i.e., except for cases of possession used to express a body part, e.g., ``my hand'', or a relative, e.g., ``my father'', which is not a Scene).
\be
\item
``[This$_E$ book$_C$]$_A$ is$_F$ John$_A$ 's$_S$''
\item
``[This$_E$ book$_C$]$_A$ is$_F$ mine$_{S+A}$'' (S+A: both an S and an A)
\item
``[[my$_S$ \rem{book}$_A$]$_E$ book$_C$]$_A$ is$_F$ red$_{S}$''
\ee
%\nss{What about `my book', `John's book'?}
\item
Existential Scenes: This is a special type of a static Scene. Since ``there are'' determines the relation here (namely existence), it is marked as S. 
\be
\item
``There$_S$ are$_F$ [thousands$_Q$ of$_R$ us$_C$]$_A$''
\ee


\textbf{Note:} the category is not defined by the words comprising the unit, but by the function the unit has in the unit it is placed in. 
Consider these pairs of examples:
\be
\item
``John$_A$ is$_F$ sitting$_P$ [in$_R$ the$_E$ garden$_C$]$_A$'' / ``[The$_E$ apple$_E$ tree$_C$]$_A$ is$_F$ in$_S$ [the$_E$ garden$_C$]$_A$''
\item
``[John$_A$ bought$_P$ wine$_A$]$_H$ for$_L$ [[Mary$_C$ 's$_R$]$_A$ birthday$_P$]$_H$'' 
/ ``[This present]$_A$ is$_F$ for$_S$ [[John$_C$ 's$_R$]$_A$ birthday$_P$]$_A$''
\ee
\ee

\paragraph{Scene or not a Scene.} One of the most important decisions in UCCA annotation is to determine whether a relation is an S/P (and evokes a Scene) or not. 
Processes are usually easier to spot -- they describe an event that evolves in time, usually some action or movement. As for States, they differ from non-Scenes 
in not being inherent properties of the Center, but something that may have been different in the past or will be different in the future.
\be
\item
``[The$_E$ outbreak$_C$]$_D$ [of$_R$ the$_E$ War$_C$ ]$_P$'' -- a Scene.
\item
``Oscillating$_P$ [between$_R$ atheism$_C$ and$_N$ agnosticism$_C$]$_A$'' -- a Scene.
\item
``[John$_C$ 's$_R$]$_A$ accurate$_D$ kick$_P$'' -- a Scene.
\item
``[[broken$_S$ \rem{glass}$_A$]$_E$ glass$_C$]$_A$ is$_F$ dangerous$_S$'' -- a Scene.
\item
``John$_A$ always$_D$ wanted$_P$ [a$_E$ garden$_C$ [ with$_S$ trees$_A$ \rem{garden}$_A$ ]$_E$]$_A$'' -- ``with trees'' is a Scene, since it is some property of the garden that could potentially change. 
\item
``The trees are$_F$ in$_S$ [the$_E$ garden$_C$]$_A$'' -- a Scene, since being in the garden is not an inherent property of the trees.
\ee

\paragraph{One Scene or two.} Where two potentially Scene-evoking relations appear in proximity to one another, the question of whether to consider them one complex Scene 
or two separate ones arises. 
It should be one Scene if the two relations are conceptually hard to separate and are similar in their participants, time, location and ground. 
It should be two Scenes if this is not the case.

\be
\item
  ``[I got home]$_H$ and$_L$ [took a shower]$_H$'' (2 Scenes with a temporal relation)
\item
  ``[it took a lot of effort]$_H$ to$_L$ [win this fight]$_H$'' (2 Scenes, with a purposive relation)
\item
  ``[he is on vacation]$_H$, [sailing a yacht near Greece]$_H$'' (2 Scenes)
%\item
%  ``[He$_A$ is$_F$ planning$_P$ [to kick the ball]$_A$ ]$_H$'' (2 Scenes) (``to kick the ball''; is a participant Scene of ``he is planning'')
%  \dd {Is 'planning' different than 'wants'?}
\item
  ``[John$_A$ eats$_P$ \rem{enthusiastically}$_D$]$_H$ and [drinks$_P$ enthusiastically$_D$ \rem{John}$_A$]$_H$'' (2 Scenes, ``eating'' and ``drinking'' are two conceptually different actions)
%\item
%  ``[The$_E$ [decline$_C$ and$_N$ death$_C$]$_C$ ]$_P$ [of$_R$ a$_E$ monarch$_C$]$_A$'' (1 Scene)
\item
``[She$_A$ [went away]$_P$]$_H$ [angry$_P$ \rem{She}$_A$]$_{H}$'' (borderline case; could be 1 Scene or 2; the two original Scenes, of her going away and of her being angry are fused into one)
\ee

\paragraph{Secondary Verb or Participant Scene.} Distinguishing between secondary verb constructions and Participant Scene constructions is done by determining whether the sentence in question refers to one or two Scenes. 
Participant Scenes correspond to cases where there are two separate Scenes, while secondary verbs correspond to the cases where there are two relations, 
one dependent (secondary, not describing a Scene in its own right, could not by itself be the P/S of a Scene) and one independent (the main relation) within the same Scene.

\be
\item
``He$_A$ demanded$_P$ [to$_R$ see$_P$ [the$_E$ manager$_C$]$_A$ \rem{He}$_A$]$_A$]'' (2 Scenes, since the demanding and the seeing are two separate Scenes which can take place in different times and locations)
\item
``He$_A$ began$_D$ kicking$_P$ [the ball]$_A$'' (one Scene, since ``began'' does not describe an action in its own right, but is dependent on the ``kicking'')
\item
``He$_A$ wants$_D$ to$_F$ kick$_P$ [the ball]$_A$'' (one Scene, since ``wants'' does not describe an action in its own right, but is dependent on the ``kicking'')
%\item
%  ``He$_A$ was$_F$ planning$_P$ [to$_R$ kick$_P$ [the ball]$_A$ \rem{He}$_A$ ]$_A$'' (two Scenes, planning and kicking are two separate actions that do not necessarily happen in the same time)
%  \dd {same question about 'wants' and 'plans}
\item
  ``He$_A$ became$_D$ [a$_E$ doctor$_C$]$_S$'' (one Scene; the becoming and him being a doctor are the same conceptual event)
\item
``He$_A$ is$_F$ known$_P$ [as$_R$ [a$_E$ doctor$_C$]$_S$ \rem{He}$_A$ ]$_A$'' (two Scenes; him being known to be something and him being a doctor)

\item
  ``[John]$_A$ said$_P$ [he$_A$ is$_F$ [a$_E$ doctor$_C$]$_S$]$_A$'' (two Scenes; John saying and him being a doctor are easy to conceptualize as two different scnes, the sentence just places them together)
 
\ee

\paragraph {Scene within Scene or two Parallel Scenes.} In order to decide whether a Scene should be included within a larger Scene we first need to ask what role it will be assigned.
If we think it is an A then we can indeed include it as an A-Scene in the larger Scene. But If we think it should be a D or T then we instead mark it separately as an H since Ds and Ts cannot be Scenes. 

\begin {itemize}
\item 
``John said [that two men were  fighting in the street]$_A$'' Scene within Scene  (``men fighting in the street'' is an A Scene in the larger Scene)
\item 
``[John usually plays soccer]$_H$ after$_L$ [he finishes his homework]$_H$'' two Parallel Scenes (If we replace ``he \ldots homework'' with a simple non-Scene unit, e.g., ``John usually plays soccer after 16:00'', then it's clear that the relation between the units is Time, but since T can't be a Scene, we mark it as an H instead).
\item 
``[You didn't do it]$_H$ [the way]$_L$ [you should have \rem{do}$_P$ \rem{it}$_A$]$_H$'' two Parallel Scenes
(``The way you should have'' relates to the manner in which ``you didn't do it'', and therefore can theoretically be referred to as a D, but since Ds can't be Scenes we mark it as an H.)
\end {itemize}

\paragraph{Verbs that can be primary or secondary.} 
%\orig {These verbs can be used either as secondary verbs (applying to the same Scene as the main verb) or as primary verbs 
%(in that case, there are two separate Scenes). This decision is context-dependent, 
%and the annotation of these verbs depends on the decision whether to annotate a single Scene or two Scenes (see criteria above)}:
There are certain verbs that in some cases will function as secondary verbs (and therefore as Ds) and in other cases as primary verbs and this depends on the context in the specific scene under question.
\be
\item
``John$_A$ remembered$_D$ to$_F$ take$_P$ [the keys]$_A$'' (context-dependent, but it's very likely that the ``remembered'' here is mostly for emphasis and therefore secondary)
\item
``John$_A$ remembered$_P$ [ [the$_E$ hike$_C$]$_P$ [with all his friends]$_A$ ]$_A$''
\item
``John$_A$ forgot$_P$ [ how$_D$ to$_F$ [ride]$_P$ [his bicycle]$_A$ ]$_A$'' (clearly the forgetting and the riding are not in the same time)
\ee

\paragraph{Secondary predicates.} A depictive or resultative should be marked separately from the main predicate as an independent parallel Scene.

\begin {enumerate}
\item 
Depictives: 

\begin{itemize}
\item
``[John$_A$ left$_P$ home$_A$]$_H$ [young$_S$ (John)$_A$]$_H$''
\item 
``[John$_A$ ate$_P$ [the food]$_A$]$_H$ [cold$_S$ (food)$_A$]$_H$''
\item
``[He$_A$ left$_P$ [the party]$_A$]$_H$ [angry$_S$ (he)$_A$]$_H$''
\end{itemize}

\item
Resultatives: 

\begin {itemize} 
\item
``[Mary$_A$ painted$_P$ [the fence]$_A$]$_H$ [blue$_S$ (fence)$_A$]$_H$''
\item
``[He$_A$ [cried himself]$_P$]$_H$ to$_L$ [sleep$_P$ (He)$_A$]$_H$''
\end {itemize} 
\end {enumerate}

%\paragraph{Seem/look/appear/sound/feel.}\label{sec:seem} These verbs are often used without specifying the experiencer of the feeling/perception. In these cases they should be treated as a G. In case where the experiencer is stated, they should be a separate Scene.

%\be
%\item
%``[The$_E$ coffee$_C$]$_A$ seems$_G$ to$_F$ be$_F$ hot$_S$''
%\item
%``[The$_E$ car$_C$]$_A$ looks$_G$ good$_S$''
%\item
%``It$_F$ seemed$_P$ [to$_R$ Mary$_C$]$_A$ [that$_R$ [the$_E$ coffee$_C$]$_A$ is$_F$ hot$_S$]$_A$''
%\item
%``It$_F$ appears$_G$ that$_F$ he$_A$ had$_F$ left$_P$ [the country]$_A$''
%\item 
%``[Your dog]$_A$ looks$_G$ like$_F$ [a$_E$ cat$_C$]$_S$''
%\ee

\paragraph{Similes.} In most cases similes should be treated as separate Scenes:
\be
\item
``[He$_A$ eats$_P$]$_H$ like$_L$ [a horse \rem{eats}$_P$]$_H$''
\ee

An exception would be when the verb does not evoke a Scene of its own (e.g. ``looks'', ``seems''. For elaboration on these verbs, see chapter \ref {sec:copula_verbless} section \ref{seem}) and is therefore considered a G. Then the whole phrase should be marked as one Scene:
\be
\item
``He$_A$ looks$_G$ like$_F$ [a horse]$_S$''
\item
``He$_A$ looks$_G$ like$_F$ he$_A$ just$_T$ saw$_P$ [a dinosaur]$_A$''
\ee




\paragraph{Cognitive events} Cognitive events (e.g. think, see, wonder) should be marked as Processes. 

\be 
\item 
``I$_A$ see$_P$ [that you both are getting along]$_A$''
\item
``I$_A$ think$_P$ [it's OK]$_A$''
\item
``I$_A$ wonder$_P$ [whether we're doing a mistake]$_A$''
\ee

\paragraph{Results of Scenes.} Results of Scenes can be Scenes in their own right. 

\be
\item
``[the$_E$ outcome$_C$]$_P$ [[of$_R$ the$_E$ trial$_C$]$_P$ ]$_A$''
%\item
%``[the$_E$ climax$_C$]$_P$ [[of$_R$ the$_E$ performance$_C$]$_P$ ]$_A$''
%    \dd {isn't the climax more of a stage in a performance than a result of it?}
\ee



%\nss{This statement doesn't make sense to me. I think you mean that deverbal nouns can simultaneously evoke a scene and be a participant in a larger scene? I don't know that it's necessarily about the agent role, because ``the car drivers are old'' would have the same analysis, right?}




\paragraph{Noun Scenes.} A noun Scene is a case when a noun-phrase serves as a Scene and the noun itself is the main relation in the Scene (the P or S). 
They should be internally analyzed as Scenes, with a P/S, As, Ds and Ts. 
In general, deverbal nouns are cases of noun Scenes, although not all noun Scenes are formed by deverbal nouns. 
%However, nominalizations may also be modified by Elaborators which are not part of the Scene (see the ``Gone with the Wind'' example below). In that case, they should be marked as Es. Determiners and other function words of the nominalizations should be included in the P/S as Elaborators as well.

\be
\item
``[[John$_C$ 's$_R$]$_A$ accurate$_D$ kick$_P$]$_A$ saved$_P$ [the game]$_A$''
\item
``[Him$_A$ destroying$_P$ [the city]$_A$ ]$_A$ was$_F$ [a$_E$ disaster$_C$]$_S$''
\item
``[[The$_E$ destruction$_C$]$_P$ [of the city]$_A$ ]$_A$ was$_F$ [a$_E$ disaster$_C$]$_S$''
\item
``[[His]$_A$ destruction$_P$ of$_F$ [the city]$_A$ ]$_A$ was$_F$ [a disaster]$_S$''
\item
``[Gone with the Wind]$_A$ is$_F$ [one$_E$ of$_R$]$_{S-}$ [Selznick$_C$ 's$_R$]$_A$ [productions$_C$]$_{-S}$''
\item
``[War$_P$]$_A$ is$_F$ imminent$_S$'


\ee

More generally, words that derive a participant from a scene are treated as scenes.

\be
\item ``[taxi$_A$ drivers$_P$]$_A$ are$_F$ usually$_D$ old$_S$''
\item ``[participants$_P$]$_A$ are$_F$ welcome$_S$''
\ee

\paragraph{Scenes with neither a P nor an S.} Some Scenes have no P or S, since it is omitted or implied. In this case, we should add them as remote units.

\be
\item
``[John bought eggs]$_H$ and$_L$ [Mary$_A$ [chewing gum]$_A$ \rem{bought}$_P$ ]$_H$''
\item
``[John$_A$ wanted$_P$ [a real life]$_A$ ]$_H$, [not$_D$ [life in a caravan]$_A$ \rem{John}$_A$ \rem{wanted}$_P$]$_H$''
%\footnote{The internal annotation of ``a real life'' and ``life in a caravan'', which are both A Scenes is omitted for brevity.\dd {There are many cases where internal annotation is omitted, maybe we can let go of this footnote?}} 
\item
``[how about]$_S$ coffee$_A$?''
\ee

%\dd {included in "Utterances"}
%\paragraph{Adjective Scenes without a noun.} 
%When we encounter an adjective Scene without a noun (e.g. `Amazing', `Great', `Splendid') we have two options: 
%\be
%\item If we can detect the missing unit from the surrounding text we add it as a Remote unit.
%\item If we are not sure what the missing Implicit unit is, we add an Implicit unit and mark the adjective as S. 
%\ee

%\begin {itemize}
%\item ``[Amazing$_S$ ! (IMPLICIT)$_A$]$_H$''
%\end {itemize}


\paragraph{Imperatives.} Imperative clauses should be marked as a Scene, with an implicit A.

\be
\item
``Stop$_P$ (IMPLICIT)$_A$ !''
\item ``Please$_F$ [take care]$_P$ [of your brother]$_A$ (IMPLICIT)$_A$''
\ee

\paragraph{Fragments.} Occasionally, a fragment of text does not describe a Scene in its own right, nor belongs to any other specific Scene. The category of such a unit, as always, is determined by its role in a super-ordinate relation it participates in (if any). By default, it's an H.
\be
\item
``[No]$_H$, [this will not stand]$_H$''
%\item
%``Thanks$_H$''
\ee

\dd {\paragraph {Thanks/Thank you.} We differentiate between two cases:

\begin {enumerate}
\item
When the Participant "I" is implicit:
In such cases "thanks" and "thank you" should be marked P and an Implicit A should be added to stand for the person thanking. Note that in the case of "thanks" if also the person being thanked isn't mentioned  then another Implicit A should be added as well.

\begin {itemize}
\item 
``[[Thank you  UNA]$_P$ [for your wonderful hospitality]$_A$ , (IMP)$_A$]$_H$``
\item
``Thanks$_P$ [for your wonderful hospitality]$_A$ (IMP)$_A$, (IMP)$_A$]$_H$]``
\item
``[[Thank you UNA]$_P$ everyone$_A$ [for coming]$_A$, (IMP)$_A$]$_H$``
\item
``[Thanks$_P$, John$_G$!, (IMP)$_A$ (IMP)$_A$]$_H$``
\item
``[Many$_Q$ thanks$_P$, (IMP)$_A$ (IMP)$_A$]$_H$``
\item 
``[Everything was absolutely great]$_H$ so$_L$ [thanks (IMP)$_A$ (IMP)$_A$]$_H$``
\end {itemize}


\item When the person who is thanking is explicitly mentioned:
In such cases  we don't mark "thank you" as one phrase but mark separately "thank" as P and "you" as A.

\begin {itemize}
\item
``I$_A$ want$_F$ to$_F$ thank$_P$ you$_A$ [for coming]$_A$``
\item 
``I$_A$ [would like UNA]$_F$ to$_F$ thank$_P$ you$_A$ [for your help]$_A$.``
\end {itemize}

\end {enumerate}}

\paragraph{Expletive it.} Sometimes ``it'' is used to take the place of the subject when there is no other A which does so. In this case it should be marked as an F.
\be
\item
``It$_F$ is$_F$ strange$_S$ [that$_R$ I$_A$ met$_P$ him$_A$ here$_A$]$_A$''
\item
``It$_F$ is$_F$ likely$_S$ [to$_R$ rain$_P$]$_A$''
\ee

\paragraph {Cooperating participants.} If two participants cooperatively participate in the same Process or perform it in an identical manner then they should be united in one A with two Cs. \nss{This only applies if they are coordinated, right? ``John and Mary played tennis'', not ``John played tennis with Mary.''}
 
\begin {itemize} 
\item
``[John$_C$ and$_N$ Mary$_C$]$_A$ went to the park``
\item
``A conversation was held [between$_R$[[the Prime Minister]$_C$ and$_N$ [the Queen]$_C$]$_C$]$_A$``
\end {itemize}


\subsection{Remotes.}

\paragraph{Two Types of verbs that take a participant Scene.} Note that some verbs with a participant Scene have a remote unit taken from the participant Scene or vice versa. Other verbs do not exhibit such behavior.
\be
\item
``I$_A$ expected$_P$ [John$_A$ to$_F$ come$_P$]$_A$''
\item
``We$_A$ agreed$_P$ [for John to give the funeral oration]$_A$''
\item
``I$_A$ persuaded$_P$ [John$_A$ to$_F$ come$_P$]$_A$''
\item
``John$_A$ promised$_P$ [to$_F$ be$_F$ better$_S$ \rem{John}$_A$]$_A$''
\ee


\paragraph{Prominent Cases of Remote Units.} A remote unit is a unit that is referenced in a Scene in which it is not contained. By convention, the remote unit should be selected to be the minimal unit that refers to the target entity (for instance, ``table'' and not ``the red table''). Several prominent cases of remote participants:
\be
\item
Coordination. The subject is often omitted.
\be \item ``[John$_A$ [had$_F$ dinner$_C$]$_P$]$_H$ and$_L$ [went$_P$ [to$_R$ bed$_C$]$_A$ \rem{John}$_A$]$_H$'' \ee
\item
Relative Scenes. The head of the elaboration is not contained in the relative Scene.
\be \item ``[The$_E$ table$_C$ [I$_A$ 'm$_F$ using$_P$ \rem{table}$_A$]$_E$]$_A$ is$_F$ too$_D$ short$_S$'' \ee
\item
Infinitives. The subject is omitted.
\be 
\item ``[Driving$_P$ [to$_F$ school$_C$]$_A$ \rem{John}$_A$]$_A$ upsets$_P$ John'' 

\item ``[to$_F$ be$_F$ expected$_P$ [to$_F$ wash$_P$ [the$_E$ car$_C$]$_A$ \rem{Mary}$_A$ ]$_A$ (IMPLICIT)$_A$ ]$_A$ infuriates$_P$ Mary$_A$''
\ee
\ee

\subsection{Secondary Relations in Scenes.}

%\be \item
%Modification by Ds and by Es. We should distinguish two types of modifying units for an A Scene.
%\be
%\item
%Ds refer to the Scene itself and express a secondary relation in that Scene:
%\be \item ``[[John$_C$ 's$_R$]$_A$ accurate$_D$ kick$_P$]$_A$ saved$_P$ [the game]$_A$'' \ee
%\item
%Es refer to Scene when it is construed as a single entity, and don't provide any information of what is happening in the Scene itself (e.g., determiners, ordinals are clear examples):
%\be \item
%``The$_E$ final$_E$ countdown$_C$''
%\item
%``[[The$_E$]$_{P-}$ accurate$_D$ [kick]$_{-P}$]$_C$ ]$_A$ saved the game''
%\item
%``[ [Ali$_C$ 's$_R$]$_A$ ]$_{C-}$ last$_E$ [ fight$_P$]$_{-C}$ ]$_A$ will surely be remembered'' (here ``last'' is an E since it tells us nothing on the fighting Scene, but only specifies which Scene it is)
%\item
%``[[The$_E$]$_{P-}$ fruitless$_D$ [negotiations$_C$]$_{-P}$ [between$_R$ John$_C$ and$_N$ Mary$_C$]$_A$]$_A$ [lasted]$_P$ [3 years]$_A$'' (can be paraphrased to ``John and Mary fruitlessly negotiated'')
%\ee
%\ee
\paragraph{Quantity Adverbs.} Adverbs of quantity such as ``just'' and ``only'' should be annotated as Ds whenever possible.
\be \item
``There$_S$ is$_F$ only$_D$ [one piece of cake]$_A$''
\item
``[The supermarket]$_A$ is$_F$ just$_D$ around$_S$ [the corner]$_A$''
\ee


\paragraph{Negation.} Negation is considered an adverbial.
\be \item
``John$_A$ did$_F$ n't$_D$ touch$_P$ [the piano]$_A$''
\item
``[John]$_A$ is$_F$ [no]$_D$ [joker]$_P$''
\ee


Some pronouns and linkers express negation on a Scene. In this case, they also serve as Ds in that scene.
    
    \be
      \item
      ``Nobody$_{A+D}$ came$_P$ [to$_R$ [the$_E$ party$_C$]$_P$]$_A$''
      \item
   	``[I$_A$ left$_P$]$_H$ without$_{L}$ [eating$_P$ [my$_S$ banana$_A$]$_A$ \rem{I}$_A$ \rem{without}$_D$]$_H$''  
    \ee

\paragraph{D in coordination.} Occasionally, several entities are connected by an N, where there is a D (usually a frequency, probability or temporal relation) which relates specifically to one of them. In this case, the proper annotation is to annotate it as a D.
\be
\item
``He$_A$ appeared$_P$ [in$_R$ [Head of the Class]$_C$, [Freddy 's Nightmares]$_C$ , [Thirtysomething ]$_C$, and$_N$ [( [for a second time]$_D$ ) [Growing Pains]$_C$ ]$_C$ ]$_A$ .''
\item
``John$_A$ is$_F$ intending$_D$ to$_F$ go$_P$ [to$_R$ [Rome$_C$, Paris$_C$ and$_N$ [perhaps$_D$ London$_C$]$_C$]$_C$ ]$_A$''.
\dout{\item
``They$_A$ treated$_P$ us$_A$ [like$_R$ people$_C$ [not$_D$ dogs$_C$ ]''
}
\ee

\paragraph {Secondary main verbs.} Sometimes the Process appears as the subject of the sentence, 
where the main verb is the secondary verb. In these cases, we still mark the secondary verb as D, and the subject as the main relation.

\be
\item
``[John$_C$ 's$_R$]$_A$ career$_P$ ended$_D$ abruptly$_D$''
\item  
``[The$_E$ race$_C$]$_P$ began$_D$ [early$_E$ in$_R$ the$_E$ morning$_C$]$_T$''
\item
``His$_A$ service$_P$ was slow$_D$''
\ee


%\paragraph{Copulas.} Some verbs in English may serve as copulas, mostly taking an adjective or adjectival phrases as arguments. 
%These are also considered Ds:
%\be
%\item ``John$_A$ stayed$_D$ awake$_S$ [all evening]$_T$''
%\item ``Mary$_A$ turned$_D$ ill$_S$''
%\ee

\paragraph{Possession.} We distinguish between two types of possession: inalienable (body parts and relatives) and alienable (owning something). Inalienable possession does not evoke a Scene, while alienable possession does. 
Possession should not be confused with the relations of ``Corresponding Profiles'' and ``Part-Whole Relations'' (see below).
\be 
\item
``John$_A$ 's$_S$ car$_A$'' (alienable)
\item
``my$_S$ car$_A$'' (alienable)
\item
``[John$_C$ 's$_R$]$_E$ hand$_C$'' (inalienable) 
\item
``[John$_C$ 's$_R$]$_E$ brother$_C$'' (inalienable) 
\item
``my$_E$ brother$_C$'' (inalienable) 
\ee

\noindent
Possession constructions can also be used to express other relations, the range of which is language-dependent.

\be
\item
``This$_E$ seat$_C$ [of$_R$ the$_E$ Knesset$_C$]$_E$'' (not a Scene)
\ee

\noindent
In the case of non-familial relations between two animates (unlike familial ones), we do consider the unit as a Scene: 
\be
\item ``[John$_A$ 's$_S$ \rem{employee}$_A$]$_E$ employee$_C$'' 
\item ``[John$_A$ 's$_S$ \rem{friend}$_A$]$_E$ friend$_C$''
\ee

\paragraph{Framing of Scenes.}
Some Scenes are wrapped in a complex preposition that frames them (e.g., ``story of'', ``rumor of'', ``belief that'). In this case, the framing noun serves a separate Scene, which takes the framed Scene as a Participant.
     
     \begin{enumerate}
     \item
       ``[the$_E$ story$_C$]$_P$ [of$_R$ [a$_E$ young$_E$ girl$_C$]$_A$ sentenced$_P$ [to$_R$ death$_P$ \rem{girl}$_A$]$_A$ ]$_A$''
     \item
       ``[the$_E$ rumor$_C$]$_P$ [of$_R$ his$_A$ retirement$_P$]$_A$''
     \item
       ``[the$_E$]$_{P-}$ strange$_D$ [belief]$_{-P}$ [that$_R$ chickens$_A$ are$_F$ immortal$_S$]$_A$''
     \end{enumerate}


\subsection{Non-Scene Units.}

\paragraph{Determiners.} Determiners should be annotated as elaborators of the noun. 
\be \item
``The$_E$ Knesset$_C$''
\item
``A$_E$ big brown dog$_C$''
\ee

\paragraph{Distinguishing between Determiners and Prepositions.}
Prepositions (which usually function as Relators) relate between two entities while Determiners (which usually function as Elaborators)  merely mark the noun phrase. 

\be
\item
It is rare to find [people with$_R$ such$_E$ wonderful personality]  
\ee
 
\paragraph{Appositions.} Appositions are cases where two consecutive units are semantically parallel and refer to the same entity. If one is a proper name and the other isn't, the first is the C, and the second is the E.
%\oa{discuss: maybe this should be two Cs}? \dd {DD: I understand the case for 2 Cs. Con argument: keeping it as is creates a unified approach in dealing  with names; in the 'titles' section we mark names as Cs and titles as Es; although these are different issues it may be easier to remember if we treat it similarly.}

\be
\item
	``John$_C$, [my history teacher]$_E$'' (apposition; {\bf This should be an E-Scene; move})
\ee


\paragraph{Extraposition.} Cases where an E does not create a contiguous stretch of text with its center. In this case, they should be marked together as a non-contiguous unit.
\be
\item
``He saw [that painting]$_{A-}$ before, [[that lovely magnificent painting]$_E$]$_{-A}$''
\item
``I met [the guy]$_{A-}$ yesterday, [[whom I first saw in the park]$_E$]$_{-A}$''
\ee

\paragraph{Fused E Scenes.} There are many constructions that resemble an E Scene construction, 
but have a somewhat different form (they don't have a clear Center). Their internal structure should look like that of a Scene:
\be \item
``[ What$_A$ I$_A$ meant$_P$]$_A$ was$_S$ [I want to have dinner]$_A$''
\item
``[ Any$_E$ recipes$_C$ [she$_A$ used$_P$ (recipes)$_A$]$_E$]$_A$ are$_F$ marked$_P$ [in$_R$ red$_C$]$_D$''
\item
``you$_A$ are$_F$ playing$_P$ [with$_R$ somebody$_A$ better$_S$ [than$_R$ you$_C$]$_A$]$_A$''
\ee

\paragraph{Numbers and Quantifiers.} They are considered Qs. The question of their scope is not addressed in the current layer of the annotation. Therefore they are considered a part of the unit adjacent to it.
\be
\item
``[All$_Q$ Greeks$_C$]$_A$ are$_F$ mortals$_S$''
\item
``[Two$_Q$ bananas$_C$]$_A$ are$_F$ lying$_P$ [on$_R$ the$_E$ table$_C$]$_A$''
\item
``Millions$_Q$ of$_R$ homes$_C$''
\ee


%\paragraph{C within C.} The question of whether to mark all the Es in a flat structure (as in ``[big$_S$]$_E$ [brown$_S$]$_E$ dogs$_C$'') or to set some order of precedence between them (as in ``[orange$_S$]$_E$ [laptop$_E$ covers$_C$]$_C$'', where it's clear that ``orange'' elaborates ``laptop covers'' and not that ``laptop'' elaborates ``orange covers''). The rule is: ``mark a C within C whenever there is an order of precedence between the Elaborators, otherwise use a flat structure''.
%\be \item
%``[big$_S$]$_E$ [brown$_S$]$_E$ dogs$_C$''
%\item
%``[orange$_S$]$_E$ [laptop$_C$ covers$_C$]$_C$''
%\item
%``[iconic$_S$]$_E$ image$_C$ [of a longhorn cow]$_E$'' - it's not clear which E precedes which, therefore we use a flat structure by default.
%\item
%``fast car mechanic'': analysis depends on whether the mechanic is fast or the car is.
%\ee

%\paragraph{Single-word Scenes within Elaborators.}
%In some cases, we first analyze a noun-phrase with Center-Elaborator relations, and then internally annotate one of its units as a Scene. If that Scene is a one-word Scene which has a remote unit that is part of the same noun-phrase, we are not required to add it (it will be filled in automatically).\oa{this is something we may want to rethink}. \dd {OK, should say that it really helps in annotation}

%\begin{itemize}
%\item
%``[Great$_S$]$_E$ [man]$_C$'' (in principle, ``man'' should have been added as a Remote unit to the E unit, but in practice for convenience we don't add it)
%\item
%``[Brown$_S$]$_E$ [dog]$_C$'' (dog should have been added as a Remote A, but can be omitted for brevity)
%\end{itemize}


%\dd {I disabled this Ordinals section since ordinals are now Quantifiers and are mentioned in the Quantifiers section}. 
%\paragraph{Ordinals.} Ordinals are words like ``first'', ``second'', ``last'' which select a member from a group according to some order. In the current coarse-grained level of annotation, we annotate them as Es.
%\be \item
%``the$_E$ first$_E$ game$_C$''
%\item
%``the$_E$ first$_E$ movie$_C$ [I$_A$ have$_F$ ever$_D$ seen$_P$]$_E$''
%\ee 

\paragraph{Comparatives/Superlatives.} Comparatives/superlatives generally evoke a static Scene. If the domain of application is explicitly mentioned (namely the set of entities the comparison applies to), it should be marked as a participant.
\be
\item
``[Jordan]$_A$ was$_F$ better$_S$ [than$_R$ James$_C$]$_A$''
\item
``[Jordan]$_A$ was$_F$ [more$_E$ beautiful$_C$]$_S$ [than$_R$ James$_C$]$_A$''
\item
``[China]$_A$ is$_F$ [the$_E$ greatest$_C$]$_S$''
\item
``[China]$_A$ is$_F$ [the$_E$ greatest$_C$]$_S$ [place on earth]$_A$''
\ee

\paragraph{Directions.} Directions should be considered as As, as they can be said to refer to an abstract location. This applies to both absolute directions (like ``north'') and relative directions (like ``away''). \nss{not sure I find this intuitive; they feel more adverbial to me. sometimes there's a clear location that can be inferred in context (``she came in/out''---presumably we have a reference point from context) but not always. E.g., ``the bird flew up'' can simply mean the bird is ascending, without an implicit source or goal.}
\be
\item
``John$_A$ told$_P$ Mary$_A$ [to$_F$ come$_P$ in$_A$ ]$_A$''
\item
``John$_A$ walked$_P$ away$_A$''
\item
``They danced$_P$ [the night]$_T$ away$_D$'' (a non-literal use)
\ee

\paragraph{Passive ``by''.} The ``by'' of the passive should be annotated as R.
\be \item
``He$_A$ is$_F$ scolded$_S$ [by$_R$ many$_C$]$_A$''
\ee
\nss{What if there's no by-phrase: is it implicit?}

\paragraph{Preposition Stranding.} In some cases, an A is missing but its preposition is in place. We mark the preposition as an A, with an R inside of it, and add a remote C:
\be \item
``The$_E$ book$_C$ [I$_A$ 'm$_F$ looking$_P$ [for$_R$ \rem{book}$_C$ ]$_A$]$_E$''
\item
``The$_E$ work$_C$ [I$_A$ [pay$_F$]$_{P-}$ most$_D$ [attention$_C$]$_{-P}$
[to$_R$ \rem{work}$_C$]$_{A}$]$_E$''
\ee


\subsection{Processes/States.}

\paragraph{Modals and Auxiliaries.} 
%If the P/S is multi-worded, it will usually contain sub-units. The main verb is the C. Other sub-units that have significant semantic input, chiefly secondary verbs, are Ds and should not be included within the P/S.\dd {I think I would begin the paragraph from here} 
Modals should invariably be annotated as secondary verbs (and therefore as Ds). This applies to ``would'' as well. Auxiliary verbs (``be'', ``have'', ``will'' and ``do''), which do not have significant semantic input in their own right\footnote{UCCA in its foundational layer does not annotate tense. Even if it did, the tense would not be considered a feature encoded on the auxiliaries, but rather in the combination of the auxiliary and the inflection.} are considered Fs.

\be 
\item
``John$_A$ will$_F$ come$_P$''
\item
``Mary$_A$ should$_D$ come$_P$''
\item
``Mary$_A$ is$_F$ coming$_P$''
\item
``John$_A$ [has to]$_D$ come$_P$''
%\dd {'to' should be taken out of the D, marked as an F, right?}\oa{I don't think so. ``have to'' is different than ``have''}\dd {OK}
\item
``I$_A$ have$_F$ done$_P$ it$_A$''
\item
``John$_A$ does$_F$ n't$_D$ know$_P$ him$_A$''
\french{
\item
``se disposait''
}
\ee

%\dd {moved it to the copulas section}:
%\paragraph{Distinguishing identity and other static Scenes.}
%Occasionally nouns are used as Ps or Ss, accompanied by some inflection of the verb ``be''. UCCA distinguishes between two cases:
%\be \item
%Identity: where there are two separate referring expressions that are asserted to have the same referent, defined entities.
%\be 
%\item
%``[The morning star]$_A$ is$_S$ [the evening star]$_A$''
%\item
%``[That person over there]$_A$ is$_S$ John$_A$''
%\ee

%\item
%Attribution: where there is one A, and the noun is used to describe some set of elements to which that A belongs to. In this case, we include the set-denoting noun in the P or S.
%\be \item
%"[his speech]$_A$ was$_F$ [an embarrassment]$_S$"
%\item
%``[Brad Pitt]$_A$ was$_F$ [a slimeball]$_S$ [as the boyfriend in the soap opera Dallas]$_A$''
%\item
%``John$_A$ is$_F$ [[six$_E$ years$_C$]$_E$ old$_C$]$_S$''
%\ee
%\ee

\paragraph{Infinitive ``to''.} By convention, when ``to'' is used as an F 
(same for ``zu'' in German), it should not be included within the process/state.

\be 
\item
``He wanted$_D$ to$_F$ come$_P$ home$_A$''
\item
``[to$_F$ kick$_P$ [a penalty shot]$_A$ [in soccer]$_A$ (IMPLICIT)$_A$]$_A$ is$_F$ exciting$_S$''
\ee



\paragraph{Light Verbs.} Cases where the verb is almost void of meaning, and most of the meaning is determined by the object. The verb is usually ``have'', ``give'', ``take'' or ``make'' (although there are other examples). Annotation: both the the light verb and the following object should be included inside the P/S. The light verb as an F and the object as a C.
\be \item
``John$_A$ [took$_F$ a$_E$ shower$_C$]$_P$''
\item
``Mary$_A$ [gave$_F$]$_{P-}$ John$_A$ [a$_E$ smile$_C$ ]$_{-P}$''
\item
``Brad$_A$ [made$_F$ a$_E$ guest$_E$ appearance$_C$]$_P$ [on$_R$ ABC$_C$]$_A$''
\ee


\paragraph{Possessive ``have'':} Whenever ``have''  carries the semantic meaning of ownership 
and precedes a concrete object (e.g. book, pen), it should be marked an S.
\be 
\item 
``John$_A$ has$_S$ [a book]$_A$'' 
\ee
But whenever ``have'' is part of a phrase describing an action or event,  then it should be marked as a ``light verb''. 
\be
\item
John$_A$ [had$_F$ a$_E$ walk$_C$]$_P$ yesterday$_T$''
\item
John$_A$ [has$_F$ problems$_C$]$_S$ 
\item
John$_A$ [has$_F$ hobbies$_C$]$_S$
\ee


\paragraph{Adjective followed by a Scene:} Analyzed as a D+P construction.
\be 
\item
``John$_A$ is$_F$ easy$_D$ to$_F$ please$_P$''
\item
``John$_A$ is$_F$ likely$_D$ to$_F$ leave$_P$''
\item
``John$_A$ is$_F$ ready$_D$ to$_F$ come$_P$''
\item
``London$_A$ is$_F$ great$_D$ for$_F$ music$_P$''
\ee
 
\paragraph{Causatives.} We view the causation word (often ``make'' or ``cause'') construction as a secondary verb.
\be \item
``John$_A$ makes$_D$ Mary$_A$ happy$_S$''
\item
``John$_A$ inspires$_D$ interest$_P$ [in$_R$ Mary$_C$]$_A$''
\item
``We just got$_D$ [our sunroom]$_A$ built$_P$ by Patio World''
\item
``Mary had$_D$ [her hair]$_A$ done$_P$''
\ee

\paragraph{Secondary Verbs with an additional role.}
Some secondary verbs introduce another role beside the roles of the main verb. An example is ``help'', ``force'' and ``permit''. Like all secondary verbs, such verbs are considered Ds. The additional participant is marked as an A in the Scene.
\be \item
``John$_A$ helped$_D$ Mary$_A$ climb$_P$ [the ladder]$_A$''
\item
``John$_A$ forced$_D$ [Mary]$_A$ to$_F$ climb$_P$ [the ladder]$_A$''
\item
``he$_A$ is$_F$ guilty$_D$ of$_F$ not$_D$ cleaning$_P$ [the dishes]$_A$''
\ee

\paragraph{Polite Forms.} Words that only serve as part of a construction for politely addressing someone are Fs.

\begin{enumerate}
\german{
\item
``Gehen$_P$ Sie$_F$ raus$_D$ !'' 
\item
``[[Sie und Ihr komischer Vogel]$_{G+A}$, raus$_P$] !'' ["you and your funny bird, out!"] (here ``Sie'' is part of the vocative)
\item
``Gehen$_P$ [Sie$_C$ und$_N$ Hans$_C$] raus$_D$ !''}
\dd {\item 
``I [would like UNA]$_F$ to thank you for all your help``
\item
``Could$_D$ you$_A$ help$_P$ me$_A$, please$_F$?``}
\end{enumerate}


%%%%%%%%%%%%%%%%%%%%%%%%%%%%%%%%%%%%%%%%%%%%%%%%%%%%%%%%%%%%%%%%%%%%%%
\subsection{Other Relations.}


\paragraph{Punctuation.} Not annotated in the current layer of UCCA (even commas).

\dd {\paragraph {Interjections.} short emotional utterances referring to the preceding or following text should be marked G:

\begin{itemize}
\item 
``[Ugh ! $_G$ that$_A$ is$_F$ gross$_S$]$_H$``
\item
``[Ouch ! $_G$ he$_A$ fell$_P$ [from his bike]$_A$]$_H$``
\item
``[Whoops ! $_G$ I$_A$ forgot$_D$ to$_F$ send$_P$ it$_A$]$_H$``
\item
``[Great$_G$ !  I$_A$ just$_T$ missed$_P$ [my ride back home]$_A$]$_H$``
\item
``[Great$_G$ ! I$_A$ 'm$_F$ so$_D$ happy$_S$]$_H$``
  \end{itemize}

An exception to this would be when an adjective utterance implicitly refers to a specific place, or a specific P/S (instead of generally expressing emotion regarding a certain Scene as a whole). Then instead of Ground it should be analyzed as a Scene of itself: 

Q: ``How was the cake ? ``
A: ``[Fantastic ! $_S$ (cake)$_A$]$_H$`` 

* since "cake" is explicitly mentioned in the nearby text, we add it as Remote, but in cases where the missing unit doesn't appear, we add an Implicit A and the adjective utterance should be S. e.g when a restaurant review opens with: "Great ! ", it probably means "great restaurant" but since "restaurant" is implicit, we annotate it: [``Great ! ``$_S$, (IMP)$_A$]$_H$

\paragraph{Fillers/Discourse Markers.} When fillers (e.g., ``oh'', ``well'') or discourse markers don't convey a meaning dimension that can be captured by UCCA's foundational layer, they should be marked as F. That is, if they are not (part of) Scenes, (part of) Scene elements or Linkers, they are Fs.

\begin{itemize}
\item 
``ummm$_F$ I$_A$ heard$_P$ [you$_A$ say$_P$ that$_A$ ]$_A$``
\item
``I$_A$ 'm$_F$ not$_D$, ah$_F$, interested$_P$``
\item
``well$_F$, this can pose a problem``
\item
``So$_F$, this is what we're going to do:``
\end{itemize}

}


\paragraph{Linkers with a single argument.}
We also allow Ls with a single argument. This usually happens if an L relates one Scene with everything that follows/precedes it, 
without there being any particular unit that the Scene relates to. Another case where we use a single argument linker is when one of its arguments is omitted.
An example would be a paragraph that starts with ``However'' that contrasts with everything that was written in the previous paragraph.

\paragraph {Elaboration of/by a Coordination.} When a certain unit relates to multiple units that carry an identical role, we unify all the multiple units under one parent unit. 

\begin {itemize}
\item
``I have [10$_Q$ [brothers$_C$ and$_N$ sisters$_C$]$_C$]$_A$`` 
\item
``Queen$_C$ of$_R$ [England$_C$ and$_N$ Canada$_C$]$_E$`` 
\item
``I may have forgotten my keys [on$_R$ [[the table]$_C$ or$_N$ couch$_C$]$_C$]$ _A$``
\end{itemize}

\paragraph{Vocatives.} Vocatives should be considered as Ground, as they are exclusively part of the speech event Scene.
If the Participant is not mentioned otherwise, add it as a remote

\be 
\item
``[John$_G$, who$_A$ is$_F$ this$_A$ ?]$_H$''
\item  ``[John$_G$, go$_P$ outside$_A$ (John)$_A$]$_H$'' (In this case, John is also a Participant in the Scene and therefore is added as a Remote)
\german{\item
``[Nein$_G$, Herr Kapitan]$_H$''}
\ee

\paragraph{Titles.} By convention, titles of names are considered Elaborators of the proper name.

\be 
\item
``I$_A$ can$_D$ 't$_D$ find$_P$ [Captain$_E$ Nemo$_C$]$_A$''
\item
``[Queen$_E$ Mary$_C$]$_A$ went$_D$ to$_F$ sleep$_P$''
%\dd {went$_D$ maybe?}
\ee

\paragraph{Focus Constructions.} Some constructions are used to emphasize one specific entity. These distinctions are generally not treated in this layer of annotation and are therefore Fs.
The difference between the examples below and existential ``there'' sentences is that here the main relation is not merely
the existence of some entity.
\be
\item
 `There$_F$ are$_F$ [lots$_Q$ of$_R$ them$_C$]$_A$ lurking$_P$ [in$_R$ the$_E$ bushes$_C$]$_A$''
\item
``It$_F$ was$_F$  John$_A$ who$_F$ wrote$_P$ [this$_E$ novel$_C$]$_A$''

\item
``There$_F$ are$_F$ earrings$_A$ on$_S$ [the$_E$ table$_C$]$_A$''
\german{
\item
``Es$_F$ gibt$_F$ Ringe$_A$ auf$_S$ [dem$_E$ Tisch$_C$]$_A$''}
\ee

\paragraph{Question Words.} Question words should be annotated with the same category as their respective component in a given answer. 

\be \item
``How$_D$ did you fix your car?''
\item
``Who$_A$ shot the sheriff?''
\item
``[Which$_E$ car$_C$]$_A$ did you buy?''
%\item
%''[Which car]$_A$ did you buy?''
\item
``Why$_H$ haven't you called me?''
\item
``When$_T$ will they arrive?''
\ee

The same applies to indirect questions:

\be
\item 
``Tell$_P$ me$_A$ [what$_A$ happened$_P$]$_A$''
\item
``I$_A$ wonder$_P$ [where$_A$ he is going]$_A$''
\ee

\noindent
Some of these words can also be used as Relative pronouns. In such cases they are not interrogative but merely 
relate the E Scene with the elaborated entity, so they should be marked as Rs.

\be
\item
``the$_E$ man$_C$ [who$_R$ was$_F$ n't$_D$ there$_{S}$ \rem{man}$_A$]$_{E}$''

\item
``the$_E$ tiger$_C$ [which$_R$ lost$_P$ [his$_S$ hair$_A$]$_A$ \rem{tiger}$_A$]$_{E}$''
\item
``the$_E$ city$_C$ [ [in which]$_R$ John$_A$ lives$_P$ \rem{city}$_A$]$_{E}$''
\ee



\paragraph{Non-contiguous Linkers.} 
%\orig{In some cases, the Linkers do not form one contiguous unit. We mark them by convention as two separate linkers and not as a non-contiguous unit. The units linked by these two linkers are the same.}
Sometimes a linkage relation is expressed by several words, which are not contiguous in the text, but evoke a single relation. We mark them by convention as two separate linkers and not as a non-contiguous unit.
\be \item
``[Either]$_L$ you buy it [or]$_L$ you don't''
\ee

\paragraph{Dates and Names.} Dates and names are treated as unanalyzable. Therefore, no sub-units should be annotated:
\be \item
``I live [in$_R$ [New York]$_C$]$_A$''
\item
``The event took place [on$_R$ [May 17th, 1832]$_C$]$_T$''
\item
``The event took place [on$_R$ [the 17th of May]$_C$]$_T$''
\ee


\paragraph{Reflexives.} Reflexives are the words that (in their primary sense) state that two participants of an event are one and the same (``himself'', ``themselves'', ``to one another'' etc.). In UCCA, we mark them as part of the P/S, which is considered unanalyzable. Note, however, that in some cases reflexives are not used in their primary sense. In these cases, they should be analyzed according to their meaning in the context.
\be \item
``John$_A$ [washed himself]$_P$''
\item
``Mary$_A$ [talked herself]$_P$ [into coming]$_A$''
\item
``[He$_C$ himself$_F$]$_A$ spoke$_P$ [to the manager]$_A$.'' (``himself'' here does not introduce a participant, but rather emphasizes that it was ``he'' and not someone else)
\item
``He did it [all$_E$ [by$_R$ himself$_C$]$_C$]$_D$'' (it's a D since the expression basically means that he did it alone)
\item
``John$_A$ [relieved himself]$_P$ [in$_R$ the$_E$ backyard$_C$]$_A$''
\item
``John$_A$ [established himself]$_P$ [as$_R$ a$_E$ lecturer$_C$]$_A$''
\german{
\item
``John hat$_F$ [sich gewaschen]$_P$''
\item
``[Studieren$_P$]$_A$ [lohnt sich]$_P$''
}
\ee

\paragraph{Complex Prepositions.} Some prepositions are multi-worded. They should be annotated as complex units (or as unanalyzable if they have no parts with significant semantic input). \textcolor{red}{In German this could be ``auf Grund'', ``an der Seite von'', ``des Weiteren'' etc.}
\be \item
``[According to]$_S$ John$_A$, [ [the$_E$ soup$_C$]$_A$ is$_F$ salty$_S$]$_A$''
\item
``Mary$_A$ is$_F$ [on top of]$_S$ [this$_E$ task$_C$]$_A$''
%\item
%``John$_A$ studies$_P$ [media$_C$ [[with$_F$ a$_F$ focus$_C$ on$_F$]$_R$ advertising$_C$]$_E$ ]$_A$''
\item
``[[later in]$_R$ 1988$_C$]$_T$, John$_A$ bought$_P$ [a$_E$ car$_C$]$_A$''
\ee


\paragraph{Frame of reference.} Some Scenes are accompanied by a background statement which explains in what way it should be construed. If the background does not refer to the same event as the Scene itself, it should be treated as a separate Scene.
\be \item
``[Under European law]$_H$, [this is an infringement]$_H$''
\item
``Historically$_H$, [governments favored city dwellers]$_H$''
\item
``[According to]$_L$ [the$_E$ figures$_C$]$_H$, [you lost a lot of money]$_H$''
\ee

\paragraph{Several Coordinated Processes/States.}
When encountering several coordinated Processes or States, each P/S should be annotated as an independent scene. 

\be 
\item 
``[John is [a businessman]$_S$]$_H$ , [politician$_S$ \rem{John}$_A$]$_H$ and$_L$ [Author$_S$ \rem{John}$_A$]$_H$''
\item
``[John$_A$ wrote$_P$ \rem{song}$_A$]$_H$, [recorded \rem{John}$_A$ \rem{song}$_A$]$_H$ and$_L$ [performed$_P$ [the$_E$ song$_C$]$_A$ \rem{John}$_A$]$_H$''
\ee


\subsection{Morphology.}

\paragraph{Inflectional and Derivational Morphology.} UCCA does not annotate them in the current layer. Therefore the word ``dogs'' has no sub-units and neither does the word ``talked''. This will be added in future layers.

\paragraph{Coersed Word/Phrase.}
Several words that were coersed into one and obtained their own idiosyncratic meaning. In this layer of UCCA they should be analyzed as a single unit, without sub-units.
\be \item
``There are pickpockets$_A$ in this side of town''
\item
``he$_A$ 's$_F$ [a$_E$ have-been$_C$]$_S$''
\item
``Let's$_D$ go$_P$ [to$_R$ the$_E$ merry-go-round$_C$]$_A$''
\ee


%%%%%%%%%%%%%%%%%%%%%%%%%%%%%%%%%%%%%%%%%%%%%%%%%%%%%%%%%%%%%%%%%%%%%%%%%%%%%%%%
\section{Chapter E: Criteria for compound splitting in German}

Some of the examples are adapted from Schulte im Walde et al., 2016. This section is co-authored with Jakob Prange and Nathan Schneider.

\paragraph{Criterion 1:} {\bf Is the compound semantically transparent or opaque?}


\be
\item
Split transparent compounds. 
\begin{itemize}
\item
The meaning of {\it Ahornblatt} (maple leaf) can be derived from the meanings of {\it Ahorn} (maple) and {\it Blatt} (leaf).
\end{itemize}

\item
Don't split opaque compounds.
\begin{itemize}
\item
The meaning of {\it Maulwurf} (mole) 	cannot be derived from the meanings of {\it Maul} (mouth of an animal) and {\it Wurf}\, (throw).
\end{itemize}

\item
Don't split partially/asymmetrically transparent compounds.

\begin{itemize}
\item
The meaning of {\it Zeitungsente} (newspaper hoax) cannot be derived from the meaning of {\it Ente} (duck), but it can be derived from the meaning of {\it Zeitung} (newspaper).
\item
{\it Murmeltier} (marmot) is a {\it Tier} (animal) but it does not involve either the noun {\it Murmel} (marble) or the verb {\it murmeln} (murmur).
\item
{\it Sonnenk\"onig} (``Sun King'', aka King Ludwig XIV) is a {\it K\"{o}nig} (king), but it doesn't involve a {\it Sonne} (sun). It's more of a name, and hence should not be split.
\item
{\it Geduldsfaden} (thread of patience) refers to the extent of one's patience, but doesn't involve a thread. Note that this is different from the metaphorical use of {\it Faden} (thread) as part of a conversation. Also, you cannot paraphrase it with {\it Faden der Geduld}, cf. Criterion 2.
\item
{\it Schriftzug} (logo) refers to something written ({\it Schrift} = writing), but it doesn't have to be an actual hand movement {\it Zug} (stroke) anymore, although it is derived from that originally.
\end{itemize}
\ee


\paragraph{Criterion 2:} {\bf Can the compound be paraphrased as a noun phrase with the same noun head?}

If it can be paraphrased, it should be split.

\begin{enumerate}
\item
{\it Kaufleute} (salesmen) $\rightarrow$ {\it Leute, die kaufen und verkaufen} (people that buy and sell).
\item
{\it Kinderbuch} (children's book) $\rightarrow$ {\it ein Buch f\"{u}r Kinder} (a book for children)
\item
{\it spindelf\"{o}rmig} (spindle-shaped) $\rightarrow$ {\it hat die Form einer Spindel} (has the shape of a spindle)
\end{enumerate}

{\bf Note:} Even if the head of the compound is a metaphor, if the same metaphor can be used in a paraphrase, the compound is considered compositional and should be split: {\it Bergkette} $\rightarrow$ {\it eine Kette von Bergen} (a chain of mountains), even though it's not an actual chain, but rather a chain-like arrangement of mountains.


\paragraph{Criterion 3:} {\bf Is the pattern of the compound productive? That is, can one or both of the words of the compound be altered, while retaining a similar meaning?}


\begin{enumerate}
\item
If it is, it should be split.
\begin{itemize}
\item
{\it Frucht{\bf saft}}, {\it Apfel{\bf saft}}, {\it Orangen{\bf saft}} (types of juice)
\item
{\it Schiffs{\bf herr}} (ship owner), {\it Haus {\bf herr}} (house owner)
\item
{\it Braun{\bf b\"{a}r}}, {\it Schwarz{\bf b\"ar}}, {\it Grizzly{\bf b\"ar}} (different species of bears);
BUT: {\it Waschb\"{a}r} (raccoon), {\it Armeisen{\bf b\"{a}r}} (anteater) should not be split.
\item
{\it Gebirgs{\bf zug}} (mountain range), {\it Sieges{\bf zug}} (triumphal march), {\it Vogel{\bf zug}} (bird migration) are all related,
BUT: {\it Schrift{\bf zug}} (logo) doesn't have much to do with the above compounds and should not be split.
\end{itemize}
\item
Where one of the words of the compound cannot be used as a free word, or has a very different meaning when used that way, it should not be split.
\begin{itemize}
\item
{\it Uhrwerk}, {\it Fachwerk}, {\it Triebwerk}, {\it Schuhwerk}, {\it Blattwerk} are all related, BUT {\it Werk} is an opus, a piece of art or a factory and therefore should not be split (borderline).
\end{itemize}
\end{enumerate}


\section{Chapter F: Possible Post-processing Notes}

\begin{itemize}
\item
  In E Scenes, put the Cs elaborated on in the Elaborator Scene.
\item
  In Ground, extract the G from the Scene they are positioned in, and add a root node whose children are the G and
  the Scene.
\item
 Flag: turn all the Rs into Fs, especially if a PSS layer is included.
\item
  Include the determiners within the main relation if they are in an A-Scene noun phrase.
\item
  Possessive pronouns should be S+A
\item
  Negative polarity relators (without, neither) should be annotated both as negation and as L/R.
\end{itemize}


\section{Chapter G: Plain Text Notation}


In order to make UCCA's annotation legible and standardized, we give here guidelines for UCCA's notation in plain text. We note that the hierarchical structure formed by UCCA can be annotated by standard bracketing. The abbreviation of the category should be either adjacent to the left or to the right side of the category.
For example, annotating the word ``apple'' with the category X should look like ``[X apple]'' or ``[apple X]''.
We use the following abbreviations for the categories:\\

\noindent
T -- time \\
Q -- quantifier\\
H -- parallel Scene \\
A -- participant \\
C -- center \\
L -- linker\\
D -- adverbial\\
E -- elaborator\\
G -- ground\\
S -- state\\
N -- connector\\
P -- process\\
R -- relator\\
F -- function\\

%\oa{Dotan -- this is plain text notation so we can't use underscores. We have to use plain text exclusively. Could you fix this?} \dd {made corrections, hope I understood correctly?}

\paragraph{Non-contiguity:} We use a dash to indicate a continuation of a unit. For example, if "word1 $\ldots$ word2" is a non-contiguous unit then we mark it ``[X- word1] [Y] [Z] [W] [-X word2]''.
%\dd {``[word1]$_{X-}$ [Y] [Z] [W] [word2]$_{-X}$".}

``[John A] [P- took] [Mary A] [up on -P] [ [her A] [promise P ] A]''


In case there are two non-contiguous units nested within one another, and of the same category, we may use indices to disambiguate. For example, in the sequence ``w1 w2 w3 w4 w5'', if ``w1 \ldots w4'' is a non-contiguous unit of category X and ``w2 \ldots w5'' is also a non-contiguous unit of category X, we mark it ``[X1- w1] [X2- w2] w3 [-X1 w4] [-X2 w5]''.


\paragraph{Remote Units:}
%\orig {To mark remote units, we first assign an index to the unit that appears as a remote unit. We mark that index immediately after the category (e.g., ``[X1 w1]''). If w1 is then a remote unit in another unit, 
%we mark it by adding the index inside curly brackets. For example, we can mark ``[X1 w1] [Y\{X1-CAT\} w2]'', meaning that X1 is a remote unit in Y of category CAT.}
%When the category of the original occurrence of the remote unit (X) is the same as the category it has in its remote occurrence, we can omit the specification of the category CAT and simply write ``[X1 w1] [Y{X1} w2]''.
We place the Remote unit inside its Parent unit at the end of the phrase in round brackets and assign it with the relevant category:
\begin{itemize}
\item
``John got home and [took a shower (John A)]''
\end{itemize}

\paragraph{Implicit Units:}
%\orig {Implicit units are marked much like remote units, but instead of writing a pointer to the remote unit, we write a fixed expression ``IMPLICIT''. For instance: ``[X1 w1] [Y{IMPLICIT-A} w2]''.}
Implicit units are marked much like remote units, the only difference is that we add a fixed expression ``IMPLICIT'' inside the round brackets. 
\begin{itemize}
\item
``[Not going there any more (IMP A)]'' (Who is not going there is implicit)
\end{itemize}


%%%%%%%%%%%%%%%%%%%%%%%%%%%%%%%%%%%%%%%%%%%%%%%%%%%%%%%%%%%%%%%%%%%%%%%%%%%%%%%%%%%%%%%%%%%%
\section{Pending Issues, Caveats and Future Work}




\begin{itemize}
   \item Annotation of redundant elements
   \item Suggestion engine, including pre-marking of Hs by heuristics.
\end{itemize}

\end{document}
